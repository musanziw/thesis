\section{Aperçu générale}
Ce présent travail consiste en la conception et le développement d'une application web d'aide à la délibération des étudiants à l'université Nouveaux Horizons. L'objectif est de faciliter la gestion des données des étudiants, la publication des résultats et de réduire le circuit de circulation des informations.

Nous parlons d'aide à la délibération des étudiants car ce processus est complexe et prend en compte plusieurs aspects tels que la conduite, les notes, la présence, etc. Cependant, dans ce travail, nous ne prenons en compte qu'un seul aspect qui est la performance académique des étudiants.

Pour ce faire, nous avons commencé par analyser la problématique et les besoins de l'université Nouveaux Horizons en matière de délibération des étudiants, afin de proposer une solution adéquate.

Nous avons ensuite effectué une étude comparative des applications existantes et des technologies utilisées pour leur développement. Nous nous sommes inspirés de leurs fonctionnalités et de leurs technologies. Nous nous sommes également intéressés à la manière dont les autres universités gèrent cette problématique.

Après une étude minutieuse des applications existantes, nous avons procédé à la conception de notre application en nous basant sur les besoins de l'université Nouveaux Horizons et en étendant les fonctionnalités des applications existantes pour ne pas réinventer la roue.

Ensuite, nous avons procédé au développement de notre application en utilisant les technologies que nous avons jugées les plus adaptées.

\section{Contexte et motivation}
Ce projet est axé sur la délibération des étudiants à l'université Nouveaux Horizons. Comme mentionné ci-dessus, la délibération est un processus qui prend en compte plusieurs aspects, et nous nous sommes focalisés sur un seul aspect qui est la performance académique de ces derniers.

Ce travail trouve tout son intérêt dans le fait que l'université Nouveaux Horizons est une institution moderne. Nous voulons donc apporter notre contribution à son essor en lui fournissant un outil qui répondra à ses besoins et qui pourra s'intégrer facilement dans son système d'information.

Nous sommes également motivés par le fait de pouvoir fournir une base qui pourra être étendue pour obtenir un système de gestion des données des étudiants plus complet et plus efficace.
\section{Problématique}
Une meilleure solution est celle qui répond à un besoin réel. Actuellement, l'université Nouveaux Horizons est confrontée à un problème de gestion des données des étudiants et de publication des résultats.

Voici un bref aperçu du circuit de circulation des informations :

\begin{itemize}
    \item Après les examens, le professeur ou le chargé de cours envoie soit les notes annuelles, soit les notes de tous les travaux ainsi que l'examen au décanat de l'université, en copie au doyen de la faculté.
    \item Le décanat, après réception, transmet les notes reçues soit au format pdf, soit Excel (xlsl), soit manuscrit au président du jury.
    \item Le jury calcule les moyennes et, après délibération, communique les résultats aux étudiants.
\end{itemize}

Vous conviendrez avec nous que le circuit de circulation des informations est assez long. De plus, il est difficile de gérer les données des étudiants de manière efficace et la publication des résultats n'est pas aussi évidente qu'elle devrait l'être.

Par ailleurs, il est difficile de suivre l'évolution des étudiants d'une année à l'autre.

Les dites données peuvent être :
\begin{itemize}
    \item les cours en compléments,
    \item le(s) relevé(s) de chaque année.
    \item etc.
\end{itemize}

\section{Méthodes et techniques}
Pour la réalisation de ce travail nous avons utilisé plusieurs méthodes et techniques qui nous
ont permis de mener à bien notre projet en nous permettant de réunir les informations nécessaires
envue d'en tirer des conclusions et de proposer la solution la mieux adaptée.

\subsection*{Méthodes}
\begin{enumerate}
    \item Méthode analytico-déductive : Nous avons analysé la problématique évoquée ci-dessus en partant des faits concrets pour aboutir à une conclusion générale.
    \item Méthode descriptive : Nous avons décrit notre problématique de manière précise et objective.
    \item Méthode comparative : Nous avons comparé la manière dont la problématique est gérée ailleurs pour en dégager les similitudes et les différences.
\end{enumerate}

\subsection*{Techniques}
Nous avons utilisé la technique documentaire, plus précisément la technique de la recherche bibliographique.

Nous avons consulté des ouvrages, des articles, des documents et des sites web pour obtenir des informations sur les applications existantes et les technologies utilisées pour leur développement. Nous avons également consulté des documents sur les méthodes de conception et de développement d'applications web.

Nous avons également utilisé la technique de l'entretien pour recueillir des informations sur les besoins du corps académique de l'université Nouveaux Horizons et sur les fonctionnalités qu'ils souhaiteraient avoir dans une application d'aide à la délibération. Nous avons également cherché à savoir comment les universités sœurs gèrent cette problématique.
\section{Etat de la question}
Étant dans une université, la délibération est un processus commun. Nous avons pensé que d'autres universités ont forcément été confrontées à la même problématique que nous.

Nous nous sommes donc intéressés à la manière dont les autres universités gèrent la délibération des étudiants et les outils qu'elles utilisent.

Nous avons trouvé des concepts qui nous ont aidé à mieux comprendre la problématique et à mieux cerner les besoins. Nous ne manquerons pas de les mentionner dans la suite de ce travail.
\section{Hypothèse}
Dans le but d'apporter une solution efficace, flexible et solide à la problématique posée, nous avons défini l'objectif de fournir une application web qui :
\begin{itemize}
    \item permettra de générer une grille de délibération,,
    \item permettra de faciliter la publication des résultats,
    \item permettra de faciliter le suivi des données des étudiants,
    \item pourra s'intégrer facilement dans le système d'information de l'université, ainsi qu'aux autres applications existantes telles que Moodle, Google Classroom, etc.
\end{itemize}

\section{Subdivision du travail}\label{sec:subdivision-du-travail}
Ce travail est subdivisé en quatre chapitres.
Le premier chapitre est consacré aux généralités,
où nous avons défini quelques concepts clés pour
une meilleure compréhension du travail. Le deuxième
chapitre est consacré au cadre du travail, où nous avons
présentons l'Université Nouveaux Horizons. Le troisième chapitre est consacré à
l'analyse, où nous présentons les différents diagrammes, enfin, le quatrième chapitre est
consacré aux choix techniques et à la présentation des résultats.