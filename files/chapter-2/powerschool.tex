\begin{enumerate}
    \item \textbf{Présentation de l'application} \newline PowerSchool est un système d'information pour les établissements scolaires qui permet de gérer les informations relatives aux étudiants, aux enseignants et aux programmes académiques. Il est utilisé par plus de 45 millions d'utilisateurs dans le monde entier, notamment dans les écoles primaires, secondaires et les établissements d'enseignement supérieur.
    \item \textbf{Fonctionalités clés} 
        \begin{itemize}
            \item Gestion des inscriptions et des admissions : PowerSchool permet de gérer les inscriptions et les admissions des étudiants, y compris la collecte des informations de base, la gestion des documents requis, la communication avec les parents et les étudiants, et la planification des cours.
            \item Gestion des notes et des absences : PowerSchool permet aux enseignants d'enregistrer les notes et les absences des étudiants, et de partager les informations avec les parents et les étudiants.
            \item Gestion des emplois du temps et des calendriers académiques : PowerSchool permet de gérer les emplois du temps des enseignants et des étudiants, ainsi que les calendriers académiques et les événements scolaires.
            \item Gestion des frais de scolarité et des paiements : PowerSchool permet de gérer les frais de scolarité et les paiements, y compris la collecte et le suivi des paiements, la génération de factures, et la communication avec les parents et les étudiants.
            \item Communication avec les parents et les étudiants : PowerSchool permet de communiquer avec les parents et les étudiants via des messages personnalisés, des bulletins électroniques, des alertes automatiques, et d'autres outils de communication.
            \item Gestion des bibliothèques : PowerSchool permet de gérer les bibliothèques de l'établissement, y compris la gestion des prêts de livres, la gestion des retours, et la gestion des réservations de livres.
            \item Rapports et analyses : PowerSchool permet de générer des rapports et des analyses sur les performances académiques des étudiants, les tendances de fréquentation, les statistiques de paiement, et bien plus encore.
        \end{itemize}
    \newpage
    \item \textbf{Inconvénients et limites}
        \begin{itemize}
            \item Coût : Le coût initial de mise en place de PowerSchool peut être élevé, en particulier pour les établissements scolaires plus petits. Les coûts peuvent également augmenter avec l'ajout de fonctionnalités supplémentaires.
            \item Complexité : PowerSchool est un système complexe qui peut nécessiter une formation pour les administrateurs, les enseignants et les utilisateurs finaux. Les établissements scolaires doivent également consacrer du temps et des ressources à la configuration et à la gestion du système.
            \item Personnalisation : Bien que PowerSchool offre de nombreuses fonctionnalités, il peut être difficile de personnaliser le système pour répondre aux besoins spécifiques de chaque établissement. Les établissements peuvent être limités dans leur capacité à personnaliser le système en fonction de leurs besoins spécifiques.
            \item Dépendance : Les établissements qui utilisent PowerSchool dépendent du système pour gérer les données des étudiants et de l'établissement. En cas de panne du système ou de perte de données, cela peut avoir des conséquences importantes pour l'établissement.
        \end{itemize}
\end{enumerate}