Dans cette partie du travail, nous allons définir quelques termes clés qui seront
utilisés tout au long du document afin d'améliorer la compréhension de celui-ci.
Cette étape est essentielle pour garantir que les lecteurs puissent comprendre
facilement les concepts et les idées présentés dans la thèse. En effet,
une bonne définition des termes clés permet de clarifier les concepts importants
et de faciliter la lecture et la compréhension du document. Par conséquent, cette partie
de la thèse est cruciale pour assurer la clarté et la précision de l'ensemble du travail.

\section{Délibération}\label{sec:deliberation}
La délibération est une confrontation de vue
visant à trancher un problème ou un choix difficile par l'adoption d'un
jugement ou d'une décision réfléchie \cite{deliberation}\cite{deliberation2}. Elle peut être effectuée par un individu seul,
mais aussi par un groupe d'individus ou une collectivité. Elle débouche en général sur
une décision ou un choix.

\section{Logiciel}\label{sec:application}
Selon la définition de l'IEEE, un logiciel est une collection de
programmes informatiques, de données et de documentation associée qui fournit des
instructions pour l'opération d'un ordinateur ou d'un système informatique, ainsi que pour la
résolution de problèmes spécifiques \cite{IEEEglossary}. Les logiciels peuvent être classés en
différentes catégories, telles que les systèmes d'exploitation, les logiciels applicatifs, les
logiciels standards, les logiciels spécifiques et les logiciels libres \cite{logiciel}.
Ces classifications sont utiles pour comprendre les différentes fonctionnalités et caractéristiques des
logiciels, ainsi que pour choisir le type de logiciel le plus adapté aux besoins spécifiques d'un projet.

\section{Web}\label{subsec:web}
Le Web, également connu sous le nom de World Wide Web (WWW),
est un système d'information en ligne qui permet l'accès à des
documents hypertexte reliés entre eux et stockés sur des serveurs à travers le monde \cite{berners1992world}.
Le Web est accessible via internet à l'aide d'un navigateur web.

\section{API}\label{subsec:api}
Une interface de programmation d’application ou interface de programmation applicative,
souvent désignée par le terme API pour « Application Programming Interface », est un ensemble
normalisé de classes, de méthodes, de fonctions et de constantes qui sert de façade par laquelle un
logiciel offre des services à d'autres logiciels \cite{richardson2007restful}\cite{fielding2000architectural}.

\section{Méthodes de développement logiciel}\label{sec:methode-de-developpement-logiciel}
Une méthode de développement logiciel est un ensemble de principes, de pratiques et de
processus qui sont utilisés pour planifier, concevoir, développer, tester et maintenir
des logiciels. Ces méthodes fournissent des approches structurées et organisées pour la
gestion de projets de développement de logiciels. Les méthodes de développement logiciel
peuvent être utilisées pour améliorer la qualité du logiciel, augmenter la productivité,
réduire les coûts et respecter les délais de livraison.

On en retient deux grandes familles : les méthodes de développement traditionnelles et
les méthodes de développement agiles.

\subsection{Méthodes de développement traditionnelles}
Les méthodes de développement de logiciels traditionnelles suivent un
processus de développement séquentiel qui se déroule généralement en phases
distinctes, comme la planification, la conception, la mise en oeuvre,
les tests et la maintenance. Le modèle en cascade (Waterfall) est un exemple
de méthode de développement traditionnelle. Dans ce modèle, chaque phase
doit être terminée avant que la suivante ne commence \cite{pressman2010software}. Les méthodes traditionnelles
sont souvent utilisées dans des projets à grande échelle avec des exigences et des
spécifications clairement définies.

Voici quelques exemples de méthodes de développement traditionnelles :

\begin{itemize}
    \item Modèle en cascade (Waterfall)
    \item Modèle en V
    \item Modèle en spirale
    \item etc.
\end{itemize}

\subsection{Méthodes de développement agiles}
Les méthodes de développement de logiciels agiles sont
basées sur des processus itératifs et incrémentaux, où les équipes
travaillent de manière collaborative et flexible pour répondre aux besoins
changeants du client \cite{martin2008clean}. Les méthodes agiles mettent l'accent sur la livraison continue
de fonctionnalités fonctionnelles, plutôt que sur la planification exhaustive et
la documentation. Le développement Agile est souvent utilisé pour des projets avec
des spécifications moins clairement définies ou pour des projets nécessitant une grande
flexibilité.

Voici quelques exemples de méthodes de développement Agile :

\begin{itemize}
    \item Scrum
    \item Kanban
    \item Extreme Programming (XP)
\end{itemize}




\section{La modélisation}\label{sec:modelisation}
La modélisation est le processus de représentation d'un système, d'un processus,
d'un objet ou d'un concept sous forme de modèle \cite{Friedenthal2014}.

Un modèle est une représentation abstraite et simplifiée d'un système réel qui
peut être utilisée pour comprendre, analyser, prédire et concevoir le système.

\section{UML (Unified Modeling Language)}\label{sec:uml}
Le Langage de Modélisation Unifié, de l'anglais Unified Modeling Language (UML),
est un langage de modélisation graphique à base de pictogrammes conçu comme une méthode
normalisée de visualisation dans les domaines du développement logiciel et en conception orientée objet \cite{Fowler2003}\cite{Booch1999}.

L'UML est une synthèse de langages de modélisation objet antérieurs : Booch, OMT, OOSE. Principalement issu des travaux d
e Grady Booch, James Rumbaugh et Ivar Jacobson, UML est à présent un standard adopté par
l'Object Management Group (OMG). UML 1.0 a été normalisé en janvier 1997; UML 2.0
a été adopté par l'OMG en juillet 2005.
La dernière version de la spécification validée par l'OMG est UML 2.5.1 (2017).

Voici les principales notations graphiques utilisées dans UML :

\section{Base de données}
Une base de données est un outil utilisé pour stocker et récupérer des
données et des informations structurées ou non structurées liées à un thème
ou à une activité\cite{GarciaMolina2019}. Les données peuvent être stockées sous une forme très structurée,
comme dans les bases de données relationnelles, ou sous une forme moins structurée,
comme dans les bases de données NoSQL. Les bases de données peuvent être centralisées
ou réparties sur plusieurs machines. Un système de gestion de base de données (SGBD)
est utilisé pour manipuler la base de données et gérer l'accès à son contenu.
Les bases de données relationnelles sont organisées en tables, tandis que les
bases de données NoSQL s'écartent du modèle relationnel en parlant des collections. Les bases de données NoSQL
sont une famille de SGBD qui s'écartent du modèle relationnel traditionnel et sont
principalement utilisées pour les systèmes distribués à grande échelle.