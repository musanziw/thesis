Pour la réalisation de ce travail nous avons 
utilisé plusieurs méthodes et techniques qui nous ont permis de mener à bien notre projet
en réunissant les informations nécessaires et en les analysant pour en tirer des conclusions.

\subsection{Méthodes}
\begin{enumerate}
    \item La méthode analytico-déductive : Nous avons analyser la problématique évoquée dans une section ci-haut en partant des faits concrets pour aboutir à une conclusion générale.
    \item La méthode descriptive : Nous avons décrit notre problématique de manière précise et objective.
    \item La méthode comparative : Nous avons eu à comparer la manière dont la problématique est gérée ailleurs pour en dégager les similitudes et les différences.
\end{enumerate}

\newpage
\subsection{Techniques}
Nous avons utilisée est la technique documentaire et
plus précisément la technique de la recherche bibliographique. Nous avons
consulté des ouvrages, des articles, des documents et des sites web
pour avoir des informations sur les applications existantes et les
technologies utilisées pour leur développement. Nous avons aussi
consulté des documents sur les méthodes de conception et de
développement d'applications web.

Nous avons aussi utilisé la technique de l'entretien pour 
avoir des informations sur les besoins du corps acédemique de 
l'Université Nouveaux Horizons et
sur les fonctionnalités qu'ils souhaiteraient avoir 
dans une application d'aide à la délibération également sur la 
manière dont les universités soeurs gérent cette problématique.
