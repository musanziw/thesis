La délibération est un processus de prise de décision qui 
implique une discussion et une réflexion approfondies sur 
un sujet ou une question donnée. C'est une méthode qui permet 
de discuter, d'examiner et d'analyser les différentes options 
et perspectives avant de prendre une décision.

Dans le contexte d'une assemblée délibérative, 
telle qu'un conseil municipal, une assemblée générale ou un
comité, la délibération se déroule généralement selon 
un processus formel qui permet à chaque membre de présenter
son point de vue et d'argumenter en faveur de ses positions.

L'objectif de la délibération est d'aboutir à une décision 
éclairée et consensuelle, qui prend en compte les différents 
points de vue et les intérêts en présence. Ce processus 
permet également de renforcer la transparence et 
la légitimité de la décision prise, en impliquant tous 
les participants dans la prise de décision.

La délibération est souvent considérée comme une méthode de
 prise de décision plus démocratique et participative que 
 d'autres approches, telles que la décision unilatérale ou le 
 vote à la majorité simple. Elle est utilisée dans de nombreux
  domaines, tels que la politique, l'éducation, les affaires, 
  et le droit, et bien d'autres pour n'en citer que ceux-ci.
