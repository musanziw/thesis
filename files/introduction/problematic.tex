Actuellement, à l'Université Nouveaux Horizons
la déliberation des étudiants se passe commme suit :

\begin{itemize}
    \item A la fin des examens le professeur chargé du cours envoie soit 
    les notes annuelles soit les notes de tous les travaux ansi que 
    l'examen au décanat de l'Université en copie au doyen de la faculté.
    \item Le décanat après réception transmet les notes réçues soit
    au format pdf soit Excel(xlsl) soit manuscrit au président du jury.
    \item Le jury à son calcul les moyennes et après déliberation
    communique les résultats aux étudiants.
\end{itemize}

Vous conviendrez avec moi que le circuit de circlulation des
informations et assez long, de plus, il est difficile de
gérer les données des étudiants de manière efficace et la publication
des résultats n'est pas aussi évidente qu'elle le devrait. 
Les dites données peuvent être : 
\begin{itemize}
    \item les cours en compléments,
    \item le(s) relevé(s) de chaque année parce qu'un 
    étudiant peut en avoir plusieurs en une année,
    \item etc.
\end{itemize}