Étant confronté à une difficulté dans la délibération des étudiants; de la récolte des cotes à la publication des résultats et le suivi des compléments des étudiants, nous avons proposé une application qui résout ce problème, par la même occasion contribuer à l'essor technologique de l'Université Nouveaux Horizons, mettre en pratique les leçons apprises tout au long du cursus académique et fournir une base pouvant être étendue à un système plus complet.

La solution proposée est une application client serveur composée d'une API Rest d'un côté pour offrir une solution facilement extensible et indépendante, de l'autre d'une interface web. 

Dans ce travail nous nous sommes limités à fournir une solution que nous pouvons ici appelé module, qui peut certes être utilisé mais qui est loin de ce qu'on aurait voulu proposer. Elle est à cheval entre un système d'apprentissage et un système de gestion des étudiants.

Ce travail est loin d'être fini et peut en permanence être amélioré, le code source est disponible sur github, quelques pistes ultérieures peuvent être l'extension à un système d'apprentissage complet avec des dépôts des travaux, de permettre aux étudiants de se connecter, suivi de la performance, etc et aussi à un système de gestion des étudiants avec la gestion des paiements, des inscriptions, etc.


En conclusion ce travail fourni une base pouvant être étendue, une solution à un problème et surtout contribue à l'essor technologique de notre université de coeur, elle permet de calculer le pourcentage, proposer des mentions, gérer les étudiants et leurs compléments, générer une proposition de relevé pouvant être modifié ainsi donc allège la tâche du jury et lui permettant de se concentrer sur un autre aspect de la délibération.