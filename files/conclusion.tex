Face aux difficultés rencontrées dans la délibération des étudiants, de la récolte des notes à la publication des résultats et au suivi des compléments des étudiants, nous avons proposé une application pour résoudre ce problème. Nous avons ainsi contribué à l'essor technologique de l'Université Nouveaux Horizons, tout en mettant en pratique les leçons apprises tout au long de notre cursus académique. Notre objectif était également de fournir une base qui pourrait être étendue à un système plus complet.

La solution que nous avons proposée est une application client-serveur composée d'une API Rest d'un côté pour offrir une solution facilement extensible et indépendante, et d'une interface web de l'autre.

Notre travail se limite à fournir une solution que nous appelons ici "module", qui peut certes être utilisée, mais qui est loin de ce que nous aurions voulu proposer. Elle se situe entre un système d'apprentissage et un système de gestion des étudiants.

Ce travail est loin d'être fini et peut être amélioré en permanence. Le code source est disponible sur GitHub. Quelques pistes pour des développements ultérieurs pourraient être l'extension à un système d'apprentissage complet avec la gestion des travaux, la possibilité pour les étudiants de se connecter, le suivi des performances, etc. et également à un système de gestion des étudiants, avec la gestion des paiements, des inscriptions, etc.

En conclusion, ce travail fournit une base pouvant être étendue, une solution à un problème et surtout contribue à l'essor technologique de notre université. Elle permet de calculer le pourcentage, de proposer des mentions, de gérer les étudiants et leurs compléments, de générer une proposition de relevé pouvant être modifiée, ce qui allège la tâche du jury et lui permet de se concentrer sur d'autres aspects de la délibération.