\section{Aperçu générale}\label{sec:apercu-generale}
Ce présent travail consiste à la conception et au développement d'une application web
d'aide à la déliberation des étudiants à l'université Nouveaux Horizons, dans le but
de faciliter la gestion des données des étudiants, la publication des résultats et
de réduire le circuit de circulation des informations.

Nous parlons d'aide à la déliberation des étudiants parce que la déliberation des étudiants
est un processus complexe et prend en compte plusieurs aspects tels que la conduite, les notes,
la présence, etc. Et poutant dans ce travail nous n'avons pris en compte qu'un
seul aspect qui est la performance académique des étudiants.

Pour ce faire, nous avons commencé par analyser la problématique et les besoins
de l'université Nouveaux Horizons en matière de déliberation des étudiants, envue de
proposer une solution adéquate.

Nous avons ensuite fait une étude comparative des applications existantes, des technologies
utilisées pour leur développement afin de nous inspirer de leurs fonctionnalités et de leurs technologies.
Nous nous sommes aussi intéressés à la manière dont les autres universités gèrent cette problématique.

Nous avons, après une étude minitieuse des applications existantes procédé à la conception
de notre application en nous basant sur les besoins de l'université Nouveaux Horizons et sur
l'extension des fonctionnalités des applications existantes pour ne pas réinventer la roue.

Nous avons ensuite procédé au développement de notre application en utilisant
les technologies que nous avons jugées les plus adaptées.

\section{Contexte et motivation}\label{sec:contexte-et-motivation}
Ce projet est axé sur la déliberation des étudiants à l'université Nouveaux Horizons,
comme mentionner ci-dessus la déliberation est un processus prend en compte plusieurs aspects
et nous nous sommes focalisés sur un seul aspect qui est la performance académique de ces derniers.


Ce travail trouve tout son essort dans le fait que l'université Nouveaux Horizons est une institution
qui se veut moderne, nous voulons donc apporter notre contribution à l'essor de cette institution
en lui fournissant un outil qui répondra à ses besoins et qui pourra s'intégrer facilement dans son système d'information.

Nous sommes aussi motivés par le fait de pouvoir fournir une base qui pourra être étendue
pour obtenir un système de gestion des données des étudiants plus complet et plus efficace.

\section{Problématique}\label{sec:problematique}
Une meilleure solution est celle qui répond à un réel besoin, l'université Nouveaux Horizons est actuellement confrontée à un problème
de gestion des données des étudiants et de publication des résultats.

Voici un bref aperçu du circuit de circulation des informations :

\begin{itemize}
    \item Après les examens le professeur ou le chargé du cours envoie soit
    les notes annuelles soit les notes de tous les travaux ansi que
    l'examen au décanat de l'Université en copie au doyen de la faculté.
    \item Le décanat après réception transmet les notes réçues soit
    au format pdf soit Excel(xlsl) soit manuscrit au président du jury.
    \item Le jury à son tour calcule les moyennes et après déliberation
    communique les résultats aux étudiants.
\end{itemize}

Vous conviendrez avec nous que le circuit de circlulation des
informations et assez long, de plus, il est difficile de
gérer les données des étudiants de manière efficace et la publication
des résultats n'est pas aussi évidente qu'elle le devrait.

Par ailleurs, il est difficile de suivre l'évolution des étudiants
d'une année à une autre.

Les dites données peuvent être :
\begin{itemize}
    \item les cours en compléments,
    \item le(s) relevé(s) de chaque année parce qu'un
    étudiant peut en avoir plusieurs en une année,
    \item etc.
\end{itemize}

\section{Méthodes et techniques}\label{sec:methode}
Pour la réalisation de ce travail nous avons utilisé plusieurs méthodes et techniques qui nous
ont permis de mener à bien notre projet en nous permettant de réunir les informations nécessaires
envue d'en tirer des conclusions et de proposer la solution la mieux adaptée.

\subsection{Méthodes}
\begin{enumerate}
    \item La méthode analytico-déductive : Nous avons analyser la problématique évoquée ci-dessus en partant des faits concrets pour aboutir à une conclusion générale.
    \item La méthode descriptive : Nous avons décrit notre problématique de manière précise et objective.
    \item La méthode comparative : Nous avons eu à comparer la manière dont la problématique est gérée ailleurs pour en dégager les similitudes et les différences.
\end{enumerate}

\subsection{Techniques}
Nous avons utilisée est la technique documentaire et
plus précisément la technique de la recherche bibliographique.

Nous avons consulté des ouvrages, des articles, des documents et des sites web
pour avoir des informations sur les applications existantes et les
technologies utilisées pour leur développement. Nous avons aussi
consulté des documents sur les méthodes de conception et de
développement d'applications web.

Nous avons aussi utilisé la technique de l'entretien pour
avoir des informations sur les besoins du corps acédemique de
l'Université Nouveaux Horizons et
sur les fonctionnalités qu'ils souhaiteraient avoir
dans une application d'aide à la délibération également sur la
manière dont les universités soeurs gérent cette problématique.

\section{Etat de la question}\label{sec:etat-de-l-art}
Etant dans une université la déliberation est un processus commun,
nous avons pensé que d'autres universités ont forcément eu à faire face à la même problématique que nous.

Nous nous sommes donc intéressés à la manière dont les autres universités
gèrent la délibération des étudiants et les outils qu'elles utilisent.

Nous avons trouvé des concepts qui nous ont aidé à mieux
comprendre la problématique et à mieux cerner les besoins.

Nous nous manquerons pas de les mentionner dans la suite de ce travail.
\section{Objectifs}\label{sec:objectifs}
En vue d'apporter une solution efficace, flexible et solide à la problématique posée, nous avons
défini l'objectif de fournir une application web qui :
\begin{itemize}
    \item permettra de racourcir le circuit de circulation des informations,
    \item permettra de faciliter la publication des résultats,
    \item permettra de faciliter le suivi des données des étudiants,
    \item peut s'intégrer facilement dans le système d'information de l'université, et aux autres applications existantes telles que moodle, Goole Classroom, etc.
\end{itemize}

\section{Subdivision du travail}\label{sec:subdivision-du-travail}
Ce travail est subdivisé en quatre chapitres, le premier chapitre est consacré aux généralités où
nous avons défini quelques concepts clés pour une meilleure compréhension du travail,
le deuxième chapitre est consacré à l'état de l'art dans cette partie du travail nous avons fait une étude de l'existant,
le troisième chapitre est consacré à la conception o\^u nous présentons les choix techniques et les différents modèles et
le quatrième chapitre est consacré à la présentation des résultats.