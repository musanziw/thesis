La délibération est un processus de prise de décision qui
implique une discussion et une réflexion approfondies sur
un sujet ou une question donnée.

C'est une méthode qui permet de discuter, d'examiner
et d'analyser les différentes options
et perspectives avant de prendre une décision.

Son objectif est d'aboutir à une décision
éclairée et consensuelle, qui prend en compte les différents
points de vue et les intérêts en présence.

Ce processus permet également de renforcer la transparence et
la légitimité de la décision prise, en impliquant tous
les participants dans la prise de décision.

Dans le contexte de l'université Nouveaux Horizons la déliberation
est un processus permettant de prendre des décisions sur les étudiants
à la fin de chaque année académique en se basant sur certains critères comme la conduite,
les notes, la participation régulière aux cours, leur conduite etc.

Eu égard, aux faits pris en compte lors d'une déliberation nous nous sommes abstenus de dire que ce présent travail
est une \say{Application de déliberation des étudiants de l'université Nouveaux Horizons} car
cela voudrait dire que nous excluons totalement la participation des acteurs de la déliberation tels que le jury, le doyen, etc.
ce qui est loin d'être le cas.

Ce présent travail s'inscrit beaucoup plus dans le cadre d'une aide à la déliberation car elle aura en charge, une propostion de relevé, la publication des résultats, la gestion des données des étudiants, etc.

Nous parlons de proposition de relevé car, en effet, c'est le jury qui a le dernier mot sur les notes des étudiants, elle aura dont la
possibilité de modifier les notes proposées par l'application en cas de besoin(péréquation).

La déliberation n'est pas seulement accès sur les notes, elle prend en compte d'autres critères comme la conduite, voilà pourquoi nous avons
préféré parler d'une application d'aide à la déliberation car elle ne prend en compte qu'un aspect de la délibération;
certainement ceci peut être étendu à un système de gestion des informations des étudiants.

\section{Contexte et motivation}\label{sec:contexte-et-motivation}
Ce projet est axé sur la déliberation des étudiants à l'université Nouveaux Horizons,
comme mentionner ci-dessus la déliberation prend en compte plusieurs aspects
et nous nous sommes focalisés sur un seul aspect qui est l'application des étudiants(leurs notes).


Ce travail trouve tout son essort dans les faits que l'université Nouveaux Horizons est une institution
qui se veut moderne et donc il en va de soit qu'elle a besoin d'un outil sur mesure, flexible et efficace
pour l'aider dans la déliberation des étudiants et la réduction du circuit de circulation des informations.

\section{Problématique}\label{sec:problematique}
Une meilleure solution est celle qui répond à un besoin réel et qui
est adaptée à l'environnement dans lequel elle est utilisée.

L'université Nouveaux Horizons est actuellement confrontée à un problème
de gestion des données des étudiants et de publication des résultats.

Voici un bref aperçu du circuit de circulation des informations :

\begin{itemize}
    \item Après les examens le professeur ou le chargé du cours envoie soit
    les notes annuelles soit les notes de tous les travaux ansi que
    l'examen au décanat de l'Université en copie au doyen de la faculté.
    \item Le décanat après réception transmet les notes réçues soit
    au format pdf soit Excel(xlsl) soit manuscrit au président du jury.
    \item Le jury à son tour calcule les moyennes et après déliberation
    communique les résultats aux étudiants.
\end{itemize}

Vous conviendrez avec moi que le circuit de circlulation des
informations et assez long, de plus, il est difficile de
gérer les données des étudiants de manière efficace et la publication
des résultats n'est pas aussi évidente qu'elle le devrait.

Les dites données peuvent être :
\begin{itemize}
    \item les cours en compléments,
    \item le(s) relevé(s) de chaque année parce qu'un
    étudiant peut en avoir plusieurs en une année,
    \item etc.
\end{itemize}

\section{Méthodes et techniques}\label{sec:methode}
Pour la réalisation de ce travail nous avons utilisé plusieurs méthodes et techniques qui nous ont permis de mener à bien notre projet
en nous permettant de réunir les informations nécessaires envue d'en tirer des conclusions et de proposer la solution la mieux adaptée.

\subsection{Méthodes}
\begin{enumerate}
    \item La méthode analytico-déductive : Nous avons analyser la problématique évoquée ci-dessus en partant des faits concrets pour aboutir à une conclusion générale.
    \item La méthode descriptive : Nous avons décrit notre problématique de manière précise et objective.
    \item La méthode comparative : Nous avons eu à comparer la manière dont la problématique est gérée ailleurs pour en dégager les similitudes et les différences.
\end{enumerate}

\subsection{Techniques}
Nous avons utilisée est la technique documentaire et
plus précisément la technique de la recherche bibliographique. Nous avons
consulté des ouvrages, des articles, des documents et des sites web
pour avoir des informations sur les applications existantes et les
technologies utilisées pour leur développement. Nous avons aussi
consulté des documents sur les méthodes de conception et de
développement d'applications web.

Nous avons aussi utilisé la technique de l'entretien pour
avoir des informations sur les besoins du corps acédemique de
l'Université Nouveaux Horizons et
sur les fonctionnalités qu'ils souhaiteraient avoir
dans une application d'aide à la délibération également sur la
manière dont les universités soeurs gérent cette problématique.

\section{Etat de la question}\label{sec:etat-de-l-art}
Etant dans une université, qui est loin d'être la seule
au pays et dans le monde, nous avons pensé que d'autres universités
ont forcément eu à faire face à la même problématique que nous.

Nous nous sommes donc intéressés à la manière dont les autres universités
gèrent la délibération des étudiants et les outils qu'elles utilisent.

Nous avons trouvé des concepts qui nous ont aidé à mieux
comprendre la problématique et à mieux cerner les besoins.

Nous nous manquerons pas de les mentionner dans la suite de ce travail.
\section{Objectifs}\label{sec:objectifs}
En vue d'apporter une solution efficace, flexible et solide à la problématique posée, nous avons
défini l'objectif de fournir une application web qui :
\begin{itemize}
    \item permettra de racourcir le circuit de circulation des informations,
    \item permettra de faciliter la publication des résultats,
    \item permettra de faciliter le suivi des données des étudiants,
    \item peut s'intégrer facilement dans le système d'information de l'université, et aux autres applications existantes telles que moodle, Goole Classroom, etc.
\end{itemize}
