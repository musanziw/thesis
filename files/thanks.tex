Je ne saurais omettre, dans ce présent travail de passer mes remerciements les plus chaleureux du plus profond de mon coeur à toutes les personnes qui, d'une manière ou d'une autre ont contribués à l'élaboration de ce mémoire. Ce travail n'est pas un résultat personnel c'est l'aboutissement d'une construction à laquelle plusieurs ont apportés une pierre angulaire. 

Je tiens particulièrement à remercier le bon Dieu pour la vie et la grâce qu'il ne cesse de m'accorder, en second lieu je remercie mon Directeur, le docteur Saint Jean Djungu pour l'accompagnement, le soutient, orientations et ses apports combien louables pour la réalisation de ce travail.

Je tiens également à remercier mon encadreur le vaillant Assistant Jonathan Kabemba pour sa bienveillance, son sérieux et ses multiples apports que je ne saurais décrire de façon nommée qui m'ont cependant été d'une aide sans précédent.

Mes sincères remerciements à toute ma promotion, les formidables personnes qui ont contribués à la réalisation de ce présent travail par des conseils, des guides, des orientations et surtout des critiques claires.

Je ne saurais finir ce travail sans remercier ma mère Yvette Lumbu pour son amour sans mesure, sa bienveillance et surtout la confiance qu'elle a toujours porté à l'égard de ma personne ; à mon père Francis Simbi pour le sacrifice consenti une preuve si grande d'amour et de bienveillance, sa rigueur qui aujourd'hui porte des fruits.