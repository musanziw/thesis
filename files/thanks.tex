Je tiens tout d'abord à exprimer mes plus sincères remerciements à toutes les personnes qui ont contribué, de près ou de loin, à l'élaboration de ce mémoire. Ce travail n'est pas le fruit d'un effort individuel, mais plutôt l'aboutissement d'une construction collective, à laquelle chacun a apporté une pierre angulaire.

Je voudrais en premier lieu remercier le Très-Haut pour la vie et la grâce qu'Il ne cesse de m'accorder. Ensuite, je tiens à exprimer ma profonde gratitude envers mon directeur de recherche, le Docteur Saint Jean Djungu, pour son accompagnement, son soutien, ses orientations et ses apports inestimables qui ont grandement contribué à la réalisation de ce travail.

Je souhaite également exprimer ma reconnaissance envers mon encadreur, l'Assistant Jonathan Kabemba, pour sa bienveillance, son sérieux et ses multiples apports, qui ont été d'une aide inestimable pour moi.

Je tiens à remercier chaleureusement l'ensemble de ma promotion, ainsi que toutes les personnes qui ont contribué à la réalisation de ce travail en me prodiguant des conseils, des guides, des orientations et surtout des critiques constructives.

Enfin, je tiens à exprimer mes remerciements les plus sincères à ma mère, Yvette Lumbu, pour son amour indéfectible, sa bienveillance et sa confiance inébranlable envers ma personne. Je souhaite également remercier mon père, Francis Simbi, pour le sacrifice consenti et son amour paternel, ainsi que pour sa rigueur qui a porté ses fruits aujourd'hui.