Dans cette partie du travail, nous allons définir quelques termes utiles qui seront utilisés dans ce document
pour une meilleure compréhension de ce dernier.

\section{La délibération}\label{sec:deliberation}
% https://fr.wikipedia.org/wiki/D%C3%A9lib%C3%A9ration
La délibération est une confrontation de vue
visant à trancher un problème ou un choix difficile par l'adoption d'un
jugement ou d'une décision réfléchie. Elle peut être effectuée par un individu seul,
mais aussi par un groupe d'individus ou une collectivité. Elle débouche en général sur
une décision ou un choix.

Compte tenu de cette définition, nous remarquerons qu'à la base il faut donc qu'il y ait
un sujet, une question ou un fait sur lequel les avis peuvent divergés et qui nécessite
une prise de décision.
\section{Un logiciel}\label{sec:application}
% livre mr djungu
Un logiciel est un ensemble de programmes informatiques qui
permettemt à un ordinateur ou un systeme informatique de réaliser une t\^ache spécifique,
il donne à ce dernier la capacité, il est destiné à resoudre un probleme précis.

Un logiciel peut classé comme système, applicatif, standard, spécifique, ou libre,

\subsection{Un logiciel système}\label{subsec:logiciel-systeme}
% https://fr.wikipedia.org/wiki/Logiciel_syst%C3%A8me
Le logiciel système est un ensemble de programmes informatiques et de bibliothèques logicielles
qui fournit un environnement permettant de créer et d'exécuter des logiciels applicatifs.
Les fonctionnalités de base d'un ordinateur telles que la manipulation des fichiers et
des périphériques sont apportées par le logiciel système. Le logiciel système est lancé avant le
logiciel applicatif et joue le rôle d'intermédiaire entre le logiciel applicatif et le matériel de
l'ordinateur.

Les logiciels systèmes ont été créés dans le but de mieux adapter les ordinateurs aux besoins des
programmeurs de logiciels applicatifs: Ils leur permettent de se concentrer sur les problèmes propres
à l'application et faire abstraction des particularités de la machine. Contrairement au logiciel
applicatif, le logiciel système est fortement dépendant de la machine. Les logiciels système offrent
des services aux logiciels applicatifs et ne sont pas exploités directement par l'usager.

Les systèmes d'exploitation, les pilotes, les langages de programmation, et les utilitaires sont
des logiciels système. L'utilisation des langages de programmation est rendue possible par divers
programmes tels que le compilateur, l'assembleur, l'éditeur de liens et le chargeur.

\subsection{Un logiciel applicatif}\label{subsec:logiciel-applicatif}
% https://fr.wikipedia.org/wiki/Application_(informatique)
Une application, un applicatif ou encore une appli, une app est, dans le domaine informatique,
un programme (ou un ensemble logiciel) directement utilisé pour réaliser une tâche, ou un ensemble de
tâches élémentaires d'un même domaine ou formant un tout

Typiquement, un éditeur de texte, un navigateur web, un lecteur multimédia, un jeu vidéo, sont des
applications. Les applications s'exécutent en utilisant les services du système d'exploitation
pour utiliser les ressources matérielles.

\subsection{Un logiciel standard}\label{subsec:logiciel-standard}
% https://fr.wikipedia.org/wiki/Produit_informatique_standard
Un produit informatique standard ou produit informatique COTS (sigle emprunté à l'expression
d'origine anglaise « commercial off-the-shelf » qui signifie : vendu sur étagère)
désigne tout produit informatique fabriqué en série et disponible dans le commerce,
non réalisé pour un projet en particulier1.

Ces produits informatiques (logiciel ou matériel) sont de plus en plus utilisés dans des
projets qui ont pour but de réduire les coûts de conception, de développement et de maintenance.

\subsection{Un logiciel spécifique}\label{subsec:logiciel-specifique}
% https://fr.wikipedia.org/wiki/Logiciel_sp%C3%A9cifique
Un logiciel spécifique est un logiciel développé sur commande à l'attention
d'un client donné, par opposition à un logiciel standard, qui est un développé sur initiative
d'un éditeur, et vendu à de nombreux clients.

Le terme anglais correspondant à logiciel spécifique est "custom software", ou "
bespoke software". Les Britanniques parlent de bespoke development pour désigner le
développement spécifique (développement d'un logiciel spécifique).

La construction d'un logiciel spécifique est une prestation de service, qui consiste à fournir
l'expertise technique et la main d'œuvre nécessaire. Les fonctionnalités, le planning de livraison,
et les conditions de paiement font l'objet d'un contrat entre le prestataire et le client.
Le consommateur est fortement impliqué dans le processus de construction et signe la réussite du travail.

La quasi-totalité des logiciels spécifiques sont des logiciels applicatifs. Les acheteurs de
logiciels spécifiques sont des moyennes et grandes entreprises.

La construction de logiciels spécifiques est pratiquée depuis les années 1960, et elle
était initialement la seule manière d'obtenir des logiciels applicatifs. En 1998 dans
l'Union européenne, 45 \% de la production de logiciels concerne des logiciels spécifiques.

\subsection{Un logiciel libre}\label{subsec:logiciel-libre}
% https://fr.wikipedia.org/wiki/Logiciel_libre
Un logiciel libre est un logiciel dont l'utilisation, l'étude, la modification et la duplication par
autrui en vue de sa diffusion sont permises, techniquement et juridiquement1, ceci afin de
garantir certaines libertés induites, dont le contrôle du programme par l'utilisateur et la
possibilité de partage entre individus.

Ces droits peuvent être simplement disponibles cas du domaine public ou bien établis par une licence,
dite « libre », basée sur le droit d'auteur. Les « licences copyleft » garantissent le maintien de ces
droits aux utilisateurs même pour les travaux dérivés.

Les logiciels libres constituent une alternative à ceux qui ne le sont pas, qualifiés de
« propriétaires » ou de « privateurs ». Ces derniers sont alors considérés par une partie de
la communauté du logiciel libre comme étant l'instrument d'un pouvoir injuste, en permettant au
développeur de contrôler l'utilisateur.

Le logiciel libre est souvent confondu à tort avec :
\begin{itemize}
    \item les gratuiciels (freewares) : un gratuiciel est un logiciel gratuit propriétaire,
    alors qu'un logiciel libre se définit par les libertés accordées à l'utilisateur.
    Si la nature du logiciel libre facilite et encourage son partage, ce qui tend à
    le rendre gratuit, elle ne s'oppose pas pour autant à sa rentabilité principalement
    via des services associés. Les rémunérations sont liées par exemple aux travaux de
    création, de développement, de mise à disposition et de soutien technique. D'un autre
    côté les logiciels gratuits ne sont pas nécessairement libres, car leur code source n'
    est pas systématiquement accessible et leur licence peut ne pas correspondre à la
    définition du logiciel libre.
    \item l’open source : le logiciel libre, selon son initiateur, est un mouvement social qui
    repose sur les principes de Liberté, Égalité, Fraternité; l’open source quant à lui,
    décrit pour la première fois dans La Cathédrale et le Bazar, s'attache aux avantages
    d'une méthode de développement au travers de la réutilisation du code source.
\end{itemize}

\subsection{Un logiciel propriétaire}\label{subsec:logiciel-proprietaire}
% https://fr.wikipedia.org/wiki/Logiciel_propri%C3%A9taire
Un logiciel propriétaire, logiciel non libre ou parfois logiciel privatif voire logiciel privateur,
est un logiciel qui ne permet pas légalement ou techniquement, ou par quelque autre moyen que ce soit,
d'exercer simultanément les quatre libertés logicielles que sont l'exécution du logiciel pour tout type
d'utilisation, l'étude de son code source (et donc l'accès à ce code source), la distribution de copies,
ainsi que la modification du code source.

\section{Le web}\label{subsec:web}
% https://fr.wikipedia.org/wiki/World_Wide_Web
Le World Wide Web, littéralement la « toile (d’araignée) mondiale », abrégé www ou le Web, la toile mondiale ou la toile,
est un système hypertexte public fonctionnant sur Internet. Le Web permet de consulter,
avec un navigateur, des pages accessibles sur des sites. L’image de la toile d’araignée vient
des hyperliens qui lient les pages web entre elles.

Le Web est une des applications d’Internet2, distincte d’autres applications comme le courrier
électronique, la visioconférence et le partage de fichiers en pair à pair. Inventé en 1989-1990 par
Tim Berners-Lee suivi de Robert Cailliau, c'est le Web qui a rendu les médias grand public attentifs
à Internet. Depuis, le Web est fréquemment confondu avec Internet; en particulier, le mot toile est
souvent utilisé dans les textes non techniques sans qu'il soit clair si l'auteur désigne le Web ou
Internet.

\section{Une API}\label{subsec:api}
% https://fr.wikipedia.org/wiki/Interface_de_programmation
Une interface de programmation d’application1 ou interface de programmation applicative,
souvent désignée par le terme API pour « Application Programming Interface », est un ensemble
normalisé de classes, de méthodes, de fonctions et de constantes qui sert de façade par laquelle un
logiciel offre des services à d'autres logiciels. Elle est offerte par une bibliothèque logicielle ou
un service web, le plus souvent accompagnée d'une description qui spécifie comment des programmes
« consommateurs » peuvent se servir des fonctionnalités du programme « fournisseur ».

On parle d'API à partir du moment où une entité informatique cherche à agir
avec ou sur un système tiers et que cette interaction se fait de manière normalisée en
respectant les contraintes d'accès définies par le système tiers. On dit alors que le système
tiers « expose une API ».

À ce titre, des interactions aussi diverses que la signature d'une fonction, une URL ou un RPC
par exemple sont parfois considérés comme des API (ou micro-API) à part entière.

Dans l'industrie contemporaine du logiciel, les applications informatiques se servent de
nombreuses interfaces de programmation, car la programmation réutilise des briques de
fonctionnalités fournies par des logiciels tiers. Cette construction par assemblage nécessite
pour le programmeur de connaître la manière d’interagir avec les autres logiciels qui dépend
de leur interface de programmation. Le programmeur n'a pas besoin de connaître les détails de
la logique interne du logiciel tiers, et celle-ci n'est pas nécessairement documentée par le
fournisseur. Seule l'API est réellement nécessaire pour utiliser le système tiers en question.

Des logiciels tels que les systèmes d'exploitation, les systèmes de gestion de base de données,
les langages de programmation ou les serveurs d'applications comportent une ou plusieurs
interfaces de programmation.

\section{Les méthodes de développement logiciel}\label{sec:methode-de-developpement-logiciel}
% https://en.wikipedia.org/wiki/Software_development_process
Dans le domaine du génie logiciel, un processus de développement logiciel est un processus de
planification et de gestion du développement logiciel. Il consiste généralement à diviser
le travail de développement de logiciels en étapes ou sous-processus plus petits, parallèles ou
séquentiels, afin d'améliorer la conception et/ou la gestion du produit. Il est également
connu sous le nom de cycle de vie du développement logiciel (SDLC). La méthodologie peut inclure
la prédéfinition de livrables et d'artefacts spécifiques qui sont créés et complétés par une équipe
de projet pour développer ou maintenir une application.

La plupart des processus de développement modernes peuvent être vaguement décrits comme agiles.

D'autres méthodologies incluent la méthode en cascade, le prototypage, le développement itératif et
incrémental, le développement en spirale, le développement rapide d'applications et la programmation
extrême.

Un "modèle" de cycle de vie est parfois considéré comme un terme plus général pour une catégorie
de méthodologies et un "processus" de développement de logiciels comme un terme plus spécifique
pour se référer à un processus spécifique choisi par une organisation spécifique. Par exemple,
il existe de nombreux processus de développement de logiciels spécifiques qui correspondent au
modèle de cycle de vie en spirale. Ce domaine est souvent considéré comme un sous-ensemble du cycle
de vie du développement des systèmes.

\subsection{Les méthodes agile}\label{subsec:methodes-agiles}
% https://en.wikipedia.org/wiki/Agile_software_development
Dans le développement de logiciels, les pratiques agiles (parfois écrites "Agile") comprennent
la découverte des besoins et l'amélioration des solutions grâce à l'effort de collaboration
d'équipes auto-organisées et interfonctionnelles avec leur(s) client(s)/utilisateur(s) finaux,
Popularisées dans le Manifeste pour le développement agile de logiciels de 2001, ces valeurs et
principes ont été dérivés et sous-tendent un large éventail de cadres de développement de logiciels,
y compris Scrum et Kanban.

Les méthodes agiles de développement de logiciels couvrent un large éventail du cycle de vie du
développement de logiciels. Certaines méthodes se concentrent sur les
pratiques (par exemple, XP, programmation pragmatique, modélisation agile),
tandis que d'autres se concentrent sur la gestion du flux de travail
(par exemple, Scrum, Kanban). Certains soutiennent les activités de spécification et
de développement des exigences (par exemple, FDD), tandis que d'autres cherchent à couvrir
l'ensemble du cycle de développement (par exemple, DSDM, RUP).

Les méthodes de développement de logiciels agiles les plus connus sont les suivantes :


\begin{enumerate}
    % https://en.wikipedia.org/wiki/Adaptive_software_development
    \item Le développement adaptatif de logiciels (DAL) est un processus de développement de
    logiciels issu des travaux de Jim Highsmith et Sam Bayer sur le développement rapide
    d'applications (RAD). Il incarne le principe selon lequel l'adaptation continue du processus au
    travail en cours est l'état normal des choses.
    Le développement adaptatif de logiciels remplace le cycle traditionnel en
    cascade par une série répétée de cycles de spéculation, de collaboration et
    d'apprentissage. Ce cycle dynamique permet un apprentissage et une adaptation
    continus à l'état émergent du projet. Les caractéristiques d'un cycle de vie ASD
    sont qu'il est axé sur la mission, basé sur les fonctionnalités, itératif,
    limité dans le temps, axé sur le risque et tolérant au changement. Comme pour
    le RAD, le DSA est également un antécédent du développement agile de logiciels.

    % https://en.wikipedia.org/wiki/Agile_software_development
    \item Le processus unifié agile (AUP) est une version simplifiée
    du processus unifié rationnel (RUP) développé par Scott Ambler.
    Il décrit une approche simple et facile à comprendre pour développer des logiciels d'application commerciale en utilisant des techniques et des concepts agiles tout en restant fidèle au RUP.
    L'AUP applique des techniques agiles, notamment le développement piloté par
    les tests (TDD), la modélisation agile (AM),
    la gestion agile du changement et le remaniement de la base de données,
    afin d'améliorer la productivité.

    %  https://en.wikipedia.org/wiki/Rapid_application_development
    \item Le développement rapide d'applications (RAD), est à la fois un terme général pour les approches de développement de
    logiciels adaptatifs et le nom de la méthode de développement rapide de James Martin.
    En général, les approches RAD du développement de logiciels mettent moins l'accent sur
    la planification et davantage sur un processus adaptatif. Les prototypes sont souvent utilisés
    en plus ou parfois même à la place des spécifications de conception.
    L'approche RAD est particulièrement bien adaptée (mais pas uniquement) au développement de
    logiciels basés sur les besoins de l'interface utilisateur. Les concepteurs d'interfaces
    utilisateur graphiques sont souvent appelés outils de développement rapide d'applications.
    D'autres approches du développement rapide comprennent les modèles adaptatif, agile, en spirale et unifié.

    % https://en.wikipedia.org/wiki/Scrum_(software_development)
    \item Scrum est un modèle léger qui aide les personnes, les équipes et les organisations à générer
    de la valeur par le biais de solutions adaptatives à des problèmes complexes.
    Il est couramment utilisé dans le développement de logiciels, mais aussi dans
    d'autres domaines tels que la recherche, les ventes, le marketing, l'éducation et
    les technologies de pointe. Il est conçu pour des équipes de dix membres ou moins qui
    divisent leur travail en objectifs à atteindre au cours d'itérations délimitées dans le temps,
    appelées sprints. Chaque sprint ne dure pas plus d'un mois et dure le plus souvent deux semaines.
    L'équipe scrum évalue les progrès réalisés lors de réunions quotidiennes limitées dans le temps,
    d'une durée maximale de 15 minutes, appelées scrums quotidiens (réunion debout).
    À la fin du sprint, l'équipe tient deux autres réunions : une revue de sprint destinée à
    présenter le travail effectué aux parties prenantes et à solliciter des commentaires, et
    une rétrospective de sprint destinée à permettre à l'équipe de réfléchir et de s'améliorer.

    % https://en.wikipedia.org/wiki/Extreme_programming
    \item La programmation extrême (XP) est une méthodologie de développement de logiciels
    visant à améliorer la qualité des logiciels et la réactivité à l'évolution des besoins des
    clients. En tant que type de développement agile de logiciels, elle préconise des versions
    fréquentes dans des cycles de développement courts, afin d'améliorer la productivité et
    d'introduire des points de contrôle permettant d'adopter les nouvelles exigences des clients.

    Parmi les autres éléments de la programmation extrême, citons : la programmation en binôme ou
    l'examen approfondi du code, les tests unitaires de l'ensemble du code, la programmation de
    fonctionnalités uniquement lorsqu'elles sont réellement nécessaires, une structure de gestion
    horizontale, la simplicité et la clarté du code, l'attente de changements dans les exigences du
    client à mesure que le temps passe et que le problème est mieux compris, ainsi qu'une communication
    fréquente avec le client et entre les programmeurs. La méthodologie tire son nom de l'idée que les
    éléments bénéfiques des pratiques traditionnelles d'ingénierie logicielle sont portés à des niveaux
    "extrêmes". Par exemple, les révisions de code sont considérées comme une pratique bénéfique;
    poussées à l'extrême, les révisions de code peuvent être continues (c'est-à-dire la pratique de la programmation en binôme).
\end{enumerate}

\subsection{Le prototypage}\label{subsec:prototypage}
% https://en.wikipedia.org/wiki/Software_prototyping
Le prototypage de logiciels est l'activité qui consiste à créer des prototypes
d'applications logicielles, c'est-à-dire des versions incomplètes du programme
logiciel en cours de développement. Il s'agit d'une activité qui peut se dérouler
dans le cadre du développement de logiciels et qui est comparable au prototypage
tel qu'il est connu dans d'autres domaines, tels que l'ingénierie mécanique ou la fabrication.

Un prototype ne simule généralement que quelques aspects du produit final et peut être
complètement différent de celui-ci.

Le prototypage présente plusieurs avantages : le concepteur et le réalisateur du
logiciel peuvent obtenir un retour d'information précieux de la part des utilisateurs
dès le début du projet. Le client et l'entrepreneur peuvent comparer si le logiciel
réalisé correspond à la spécification du logiciel, selon laquelle le programme
logiciel est construit. Cela permet également à l'ingénieur logiciel de se faire
une idée de l'exactitude des estimations initiales du projet et de savoir si les délais
et les étapes proposés peuvent être respectés. Le degré d'exhaustivité et les techniques
utilisées dans le prototypage ont fait l'objet de développements et de débats depuis leur
proposition au début des années 1970.

\subsection{Le développement itératif et incrémental}\label{subsec:developpement-iteratif-et-incremental}
% https://en.wikipedia.org/wiki/Iterative_and_incremental_development
Le développement itératif et incrémental est une combinaison de la conception itérative ou de la méthode
itérative et du modèle de construction incrémental pour le développement.

L'utilisation de ce terme a débuté dans le domaine du développement de logiciels, une combinaison de
longue date des deux termes itératif et incrémental ayant été largement suggérée pour les efforts de
développement de grande envergure. Par exemple, le document DOD-STD-2167 de 1985 mentionne :
"Pendant le développement d'un logiciel, plus d'une itération du cycle de développement du logiciel
peut être en cours en même temps" et "Ce processus peut être décrit comme une 'acquisition évolutive'
ou une approche de 'construction incrémentale'". Dans le domaine des logiciels, la relation entre les
itérations et les incréments est déterminée par le processus global de développement du logiciel.

\subsection{Le développement en spirale}\label{subsec:developpement-en-spirale}
% https://en.wikipedia.org/wiki/Spiral_model

Le modèle en spirale est un modèle de processus de développement de logiciels axé sur le risque.
Basé sur les risques propres à un projet donné, le modèle en spirale guide l'équipe vers
l'adoption d'éléments d'un ou de plusieurs modèles de processus, tels que le prototypage incrémentiel,
en cascade ou évolutif.

\subsection{La méthode en V}\label{subsec:methode-en-v}
% https://fr.wikipedia.org/wiki/Cycle_en_V
Le cycle en V (« V model » ou « Vee model » en anglais) est un modèle d'organisation des
activités de développement d'un produit qui se caractérise par un flux d'activité descendant
qui détaille le produit jusqu'à sa réalisation, et un flux ascendant, qui assemble le produit
en vérifiant sa qualité. Ce modèle est issu du modèle en cascade dont il reprend l'approche
séquentielle et linéaire de phases.

Il l'enrichit cependant d'activités d'intégration de système à partir de composants plus
élémentaires, et il met en regard chaque phase de production successive avec sa phase de
validation correspondante, lui conférant ainsi la forme d'un V1.

Issu de l'ingénierie système, le cycle en V est souvent considéré comme un cycle de projet,
alors qu'ingénierie système et gestion de projet sont complémentaires. L'ingénierie système va se
focaliser sur le développement du produit, alors que la gestion de projet va se concentrer sur
l'atteinte des bénéfices attendus par le client ou l'utilisateur. Le cycle en V n'est donc pas un
cycle de projet.

\subsection{Le développement en cascade}\label{subsec:developpement-en-cascade}
%  https://fr.wikipedia.org/wiki/Mod%C3%A8le_en_cascade
Le modèle en cascade, ou « waterfall » en anglais, est une organisation des activités d'un projet
sous forme de phases linéaires et séquentielles, où chaque phase correspond à une spécialisation
des tâches et dépend des résultats de la phase précédente. Il comprend les phases d'exigences,
de conception, de mise en œuvre et de mise en service.

Le modèle en cascade est un cycle de vie de projet issu des industries manufacturières et du secteur
de la construction, où une conception préalable est nécessaire, compte tenu des fortes contraintes
matérielles et des coûts élevés afférents aux changements de la conception en cours de réalisation.
Il est utilisé notamment dans les domaines de l'ingénierie et du développement de logiciels.

\section{Git}\label{sec:git-1}
% https://fr.wikipedia.org/wiki/Git
Git est un logiciel de gestion de versions de code source décentralisé. C'est un logiciel libre et gratuit,
créé en 2005 par Linus Torvalds, auteur du noyau Linux, et distribué selon les termes de la
licence publique générale GNU version 2. Le principal contributeur actuel de Git, et ce
depuis plus de 16 ans, est Junio C Hamano.

Depuis les années 2010, il s’agit du logiciel de gestion de versions le plus populaire dans
le développement logiciel et web, qui est utilisé par des dizaines de millions de personnes,
sur tous les environnements (Windows, Mac, Linux).

Git est aussi le système à la base du célèbre site web GitHub,
le plus important hébergeur de code informatique.

\section{GitHub}\label{sec:github-1}
% https://fr.wikipedia.org/wiki/GitHub
GitHub est un service web d'hébergement et de gestion de développement de logiciels,
utilisant le logiciel de gestion de versions Git.
Ce site est développé en Ruby on Rails et Erlang par Chris Wanstrath, PJ Hyett et
Tom Preston-Werner. GitHub propose des comptes professionnels payants,
ainsi que des comptes gratuits pour les projets de logiciels libres.

Le site assure également un contrôle d'accès et des fonctionnalités destinées à
la collaboration comme le suivi des bugs, les demandes de fonctionnalités, la gestion
de tâches et un wiki pour chaque projet. Le site est devenu le plus important dépôt de
code au monde, utilisé comme dépôt public de projets libres ou dépôt privé d'entreprises.

En 2018, GitHub est acquis par Microsoft pour 7,5 milliards de dollars.

\section{Linux}\label{sec:linux}
% https://fr.wikipedia.org/wiki/Linux
Linux ou GNU/Linux est une famille de systèmes d'exploitation open source de type
Unix fondés sur le noyau Linux créé en 1991 par Linus Torvalds. De nombreuses distributions
Linux ont depuis vu le jour et constituent un important vecteur de popularisation du mouvement
du logiciel libre.

Si, à l'origine, Linux a été développé pour les ordinateurs compatibles PC, il n'a jamais équipé
qu'une très faible part des ordinateurs personnels. Mais le noyau Linux, accompagné ou non des logiciels
GNU, est également utilisé par d'autres types de systèmes informatiques, notamment les serveurs,
téléphones portables, systèmes embarqués ou encore superordinateurs. Le système d'exploitation pour
téléphones portables Android qui utilise le noyau Linux mais pas GNU équipe aujourd'hui
85 \% des tablettes tactiles et smartphones.

\section{La modélisation}\label{sec:modelisation}
% Livre mr djungu
La modélisation est le processus de représentation d'un système,
d'un processus, d'un objet ou d'un concept sous forme de modèle.

Un modèle est une représentation abstraite et simplifiée d'un système réel,
qui peut être utilisée pour comprendre, analyser, prédire et concevoir le système.

\section{UML (Unified Modeling Language)}\label{sec:uml}
% https://fr.wikipedia.org/wiki/UML_(informatique)
Le Langage de Modélisation Unifié, de l'anglais Unified Modeling Language (UML),
est un langage de modélisation graphique à base de pictogrammes conçu comme une méthode
normalisée de visualisation dans les domaines du développement logiciel et en conception orientée objet.

L'UML est une synthèse de langages de modélisation objet antérieurs : Booch, OMT, OOSE. Principalement issu des travaux d
e Grady Booch, James Rumbaugh et Ivar Jacobson, UML est à présent un standard adopté par
l'Object Management Group (OMG). UML 1.0 a été normalisé en janvier 1997; UML 2.0
a été adopté par l'OMG en juillet 2005.
La dernière version de la spécification validée par l'OMG est UML 2.5.1 (2017).

Voici les principales notations graphiques utilisées dans UML :

\subsection{le diagramme de classes}\label{subsec:diagrammes-de-classe}
% https://fr.wikipedia.org/wiki/Diagramme_de_classes
Le diagramme de classes est un schéma utilisé en génie logiciel pour
présenter les classes et les interfaces des systèmes ainsi que leurs relations.
Ce diagramme fait partie de la partie statique d'UML, ne s'intéressant pas aux aspects
temporels et dynamiques.

Une classe décrit les responsabilités, le comportement et le type d'un ensemble d'objets.
Les éléments de cet ensemble sont les instances de la classe.

Une classe est un ensemble de fonctions et de données (attributs) qui sont
liées ensemble par un champ sémantique. Les classes sont utilisées dans la programmation
orientée objet. Elles permettent de modéliser un programme et ainsi de découper une tâche
complexe en plusieurs petits travaux simples.

Les classes peuvent être reliées grâce au mécanisme d'héritage qui permet de mettre en évidence
des relations de parenté. D'autres relations sont possibles entre des classes, représentées par
un arc spécifique dans le diagramme de classes.

Elles sont finalement instanciées pour créer des objets
(une classe est un moule à objet : elle décrit les caractéristiques des objets,
les objets contiennent leurs valeurs propres pour chacune de ces caractéristiques
lorsqu'ils sont instanciés).

\subsection{Le diagramme de cas d'utilisation}\label{subsec:diagrammes-de-cas-utilisation}
% https://fr.wikipedia.org/wiki/Diagramme_de_cas_d%27utilisation
Les diagrammes de cas d'utilisation (DCU) sont des diagrammes UML utilisés pour une
représentation du comportement fonctionnel d'un système logiciel. Ils sont utiles
pour des présentations auprès de la direction ou des acteurs d'un projet.
En effet, un cas d'utilisation (use cases) représente une unité discrète d'interaction
entre un utilisateur (humain ou machine) et un système. Ainsi, dans un diagramme de
cas d'utilisation, les utilisateurs sont appelés acteurs (actors), et ils
apparaissent dans les cas d'utilisation.

\subsection{le diagramme d'objets}\label{subsec:diagrammes-d-objets}
% https://fr.wikipedia.org/wiki/Diagramme_d%27objets
Le diagramme d'objets, dans le langage de modélisation de donnée UML, permet de représenter
les instances des classes, c'est-à-dire des objets. Comme le diagramme de classes, il exprime
les relations qui existent entre les objets, mais aussi l'état des objets, ce qui permet d'exprimer
des contextes d'exécution. En ce sens, ce diagramme est moins général que le diagramme de classes.

Les diagrammes d'objets s'utilisent pour montrer l'état des instances d'objet avant et après une
interaction, autrement dit c'est une photographie à un instant précis des attributs et objet existant.
Il est utilisé en phase exploratoire.

\subsection{le diagramme de séquence}\label{subsec:diagrammes-de-sequence}
% https://fr.wikipedia.org/wiki/Diagramme_de_s%C3%A9quence
Les diagrammes de séquences sont la représentation graphique des interactions entre
les acteurs et le système selon un ordre chronologique dans la formulation Unified Modeling Language.

\subsection{le diagramme d'activités}\label{subsec:diagrammes-d-activites}
% https://fr.wikipedia.org/wiki/Diagramme_d%27activit%C3%A9
Le diagramme d'activité est un diagramme comportemental d'UML, permettant de représenter
le déclenchement d'événements en fonction des états du système et de modéliser des
comportements parallélisables (multi-threads ou multi-processus).
Le diagramme d'activité est également utilisé pour décrire un flux de travail (workflow).

\subsection{le diagramme de déploiement}\label{subsec:diagrammes-de-deploiement}
% https://fr.wikipedia.org/wiki/Diagramme_de_d%C3%A9ploiement
En UML, un diagramme de déploiement est une vue statique qui sert à représenter
l'utilisation de l'infrastructure physique par le système et la manière dont
les composants du système sont répartis ainsi que leurs relations entre eux.
Les éléments utilisés par un diagramme de déploiement sont principalement les noeuds,
les composants, les associations et les artefacts.
Les caractéristiques des ressources matérielles physiques et
des supports de communication peuvent être précisées par stéréotype.

\section{L'HTML (HyperText Markup Language)}\label{sec:html}
% https://fr.wikipedia.org/wiki/Hypertext_Markup_Language
Le HyperText Markup Language, généralement abrégé HTML ou, dans sa dernière version, HTML5,
est le langage de balisage conçu pour représenter les pages web.

Ce langage permet d’écrire de l’hypertexte (d’où son nom), de structurer sémantiquement
une page web, de mettre en forme du contenu, de créer des formulaires de saisie ou encore
d’inclure des ressources multimédias dont des images, des vidéos, et des programmes informatiques.
L'HTML offre également la possibilité de créer des documents interopérables avec des équipements
très variés et conformément aux exigences de l’accessibilité du web.

Il est souvent utilisé conjointement avec le langage de programmation JavaScript et des
feuilles de style en cascade (CSS). HTML est inspiré du Standard Generalized Markup Language (SGML).
Il s'agit d'un format ouvert.

\section{CSS (Cascading Style Sheets)}\label{sec:css}
% https://fr.wikipedia.org/wiki/Feuilles_de_style_en_cascade
Les feuilles de style en cascade, généralement appelées CSS de l'anglais Cascading Style Sheets,
forment un langage informatique qui décrit la présentation des documents HTML et XML.
Les standards définissant CSS sont publiés par le World Wide Web Consortium (W3C).
Introduit au milieu des années 1990, CSS devient couramment utilisé dans la conception
de sites web et bien pris en charge par les navigateurs web dans les années 2000.

Le css sert à donner vie à une page web, il permet de mettre en forme le contenu HTML
d'une page web. Il permet de définir des styles pour chaque élément HTML et de les
appliquer à l'ensemble des pages d'un site.

\section{Un language de programmation}\label{sec:language-de-programmation}
% https://fr.wikipedia.org/wiki/Langage_de_programmation
Un langage de programmation est un langage informatique destiné à formuler des algorithmes et
produire des programmes informatiques qui les appliquent. D'une manière similaire à une langue
naturelle, un langage de programmation est composé d'un alphabet, d'un vocabulaire, de règles
de grammaire, de significations, mais aussi d'un environnement de traduction censé rendre
sa syntaxe compréhensible par la machine.

Les langages de programmation permettent de décrire d'une part les structures des données
qui seront manipulées par l'appareil informatique, et d'autre part d'indiquer comment sont
effectuées les manipulations, selon quels algorithmes. Ils servent de moyens de communication
par lesquels le programmeur communique avec l'ordinateur, mais aussi avec d'autres programmeurs;
les programmes étant d'ordinaire écrits, lus, compris et modifiés par une équipe de programmeurs.


Les langages de programmation peuvent être classés en plusieurs catégories, telles que :

\begin{enumerate}
    \item Les langages de programmation impératifs : qui décrivent les étapes à suivre pour
    accomplir une tâche. Les langages de programmation impératifs peuvent être classés en
    deux catégories : les langages de programmation procéduraux et les langages de programmation
    orientés objet.
    \item Les langages de programmation fonctionnels : qui se concentrent sur les fonctions
    mathématiques et l'évaluation d'expressions, etc.
    \item Les langages de programmation orientés objet : qui sont basés sur des objets et
    leurs interactions,  etc. Les langages de programmation orientés objet peuvent être classés
    en deux catégories : les langages de programmation orientés objet basés sur les classes et
    les langages de programmation orientés objet basés sur les prototypes.
    \item Les langages de script : qui sont utilisés pour automatiser des tâches et des processus, etc.
    Les langages de script sont généralement interprétés plutôt que compilés.
    \item Les langages de programmation déclaratifs : qui décrivent le résultat souhaité plutôt que
    les étapes à suivre pour l'obtenir.
\end{enumerate}

Il existe de nombreux langages de programmation différents, chacun ayant ses propres avantages et
inconvénients. Voici quelques-uns des langages de programmation les plus courants :
\begin{itemize}
    \item C++,
    \item JavaScript,
    \item Java,
    \item CSharp,
    \item Ruby,
    \item Python,
    \item Etc.
\end{itemize}

\section{Un framework}\label{sec:framework}
% https://fr.wikipedia.org/wiki/Framework
En programmation informatique, un framework (appelé aussi infrastructure logicielle,
infrastructure de développement, environnement de développement, socle d'applications,
cadre d'applications ou cadriciel) est un ensemble cohérent de composants logiciels
structurels qui sert à créer les fondations ainsi que les grandes lignes de tout ou partie
d'un logiciel, c'est-à-dire une architecture.

Un framework se distingue d'une simple bibliothèque logicielle principalement,
d'une part par son caractère générique, faiblement spécialisé, contrairement à
certaines bibliothèques; un framework peut à ce titre être constitué de plusieurs
bibliothèques, chacune spécialisée dans un domaine.

Un framework peut néanmoins être spécialisé dans un langage particulier,
une plateforme spécifique, un domaine particulier : communication de données,
data mapping, etc.. D'autre part, il impose un cadre de travail, dû à sa construction même,
 guidant l'architecture logicielle voire conduisant le développeur à respecter certains
 patrons de conception ; les bibliothèques le constituant sont alors organisées selon le même paradigme.

Les frameworks sont donc conçus et utilisés pour modeler l'architecture des logiciels applicatifs,
des applications web, des middlewares et des composants logiciels. Les frameworks sont
acquis par les informaticiens, puis incorporés dans des logiciels applicatifs mis sur
le marché, ils sont par conséquent rarement achetés et installés séparément par un utilisateur final.

\section{NodeJS}\label{sec:nodejs}
% https://fr.wikipedia.org/wiki/Node.js
Node.js est une plateforme logicielle libre en JavaScript, orientée vers les applications
réseau évènementielles hautement concurrentes qui doivent pouvoir monter en charge.

Elle utilise la machine virtuelle V8, la bibliothèque libuv pour sa boucle d'évènements,
et implémente sous licence MIT les spécifications CommonJS.

Parmi les modules natifs de Node.js, on retrouve http qui permet le développement de
serveur HTTP. Ce qui autorise, lors du déploiement de sites internet et d'applications
web développés avec Node.js, de ne pas installer et utiliser des serveurs webs tels que Nginx ou Apache.

Concrètement, Node.js est un environnement bas niveau permettant l'exécution de JavaScript côté serveur.

Node.js est utilisé notamment comme plateforme de serveur Web, elle est utilisée par GoDaddy, IBM,
Netflix, Amazon Web Services, Groupon, Vivaldi, SAP, LinkedIn, Microsoft, Yahoo!,
Walmart, Rakuten, Sage et PayPal.

\section{Une base des données}\label{sec:base-de-donnees}
% https://fr.wikipedia.org/wiki/Base_de_donn%C3%A9es
Une base de données permet de stocker et de retrouver des données structurées,
semi-structurées ou des données brutes ou de l'information, souvent en
rapport avec un thème ou une activité; celles-ci peuvent être de
natures différentes et plus ou moins reliées entre elles.

Leurs données peuvent être stockées sous une forme très structurée
(base de données relationnelles par exemple), ou bien sous la forme de données brutes
peu structurées (avec les bases de données NoSQL par exemple). Une base de données peut
être localisée dans un même lieu et sur un même support informatisé, ou répartie sur
plusieurs machines à plusieurs endroits.

La base de données est au centre des dispositifs informatiques de collecte, mise en forme,
stockage et utilisation d'informations. Le dispositif comporte un système de gestion de
base de données (abréviation : SGBD) : un logiciel moteur qui manipule la base de données et
dirige l'accès à son contenu. De tels dispositifs comportent également des logiciels applicatifs, et
un ensemble de règles relatives à l'accès et l'utilisation des informations.

Lorsque plusieurs objets nommés « bases de données » sont constitués sous forme de collection,
on parle alors d'une banque de données.

Il existe plusieurs types de bases de données comme les bases des données en mémoire, les plus courantes sont les bases de données
relationnelles et les bases de données NoSQL.

\subsection{Les bases de données relationnelles}\label{subsec:base-de-donnees-relationnelles}
% https://fr.wikipedia.org/wiki/Base_de_donn%C3%A9es_relationnelle
Une base de données relationnelle est une base de données où l'information est
organisée dans des tableaux à deux dimensions appelés des relations ou tables, selon
le modèle introduit par Edgar F. Codd en 1960. Selon ce modèle relationnel, une base de
données consiste en une ou plusieurs relations. Les lignes de ces relations sont appelées
des nuplets ou enregistrements. Les colonnes sont appelées des attributs.

Les logiciels qui permettent de créer, utiliser et maintenir des bases de données relationnelles
sont des systèmes de gestion de bases de données relationnelles (SGBDR).

Pratiquement tous les systèmes relationnels utilisent le langage SQL pour interroger les bases de
données. Ce langage permet de demander des opérations d'algèbre relationnelle telles que l'intersection,
la sélection et la jointure.

\subsection{Les bases de données NoSQL}\label{subsec:base-de-donnees-nosql}
% https://fr.wikipedia.org/wiki/NoSQL
NoSQL désigne une famille de systèmes de gestion de base de données (SGBD) qui s'écarte
du paradigme classique des bases relationnelles. L'explicitation la plus populaire de
l'acronyme est Not only SQL (« pas seulement SQL » en anglais) même si cette
interprétation peut être discutée.

La définition exacte de la famille des SGBD NoSQL reste sujette à débat.
Le terme se rattache autant à des caractéristiques techniques qu'à une génération
historique de SGBD qui a émergé autour des années 20102. D'après Pramod J. Sadalage et Martin Fowler,
la raison principale de l'émergence et de l'adoption des SGBD NoSQL serait le développement des
centres de données et la nécessité de posséder un paradigme de bases de données adapté à ce modèle
d'infrastructure matérielle.

L'architecture machine en clusters induit une structure logicielle distribuée fonctionnant
avec des agrégats répartis sur différents serveurs permettant des accès et modifications
concurrentes mais imposant également de remettre en cause de nombreux fondements de
l'architecture SGBD relationnelle traditionnelle, notamment les propriétés ACID.
