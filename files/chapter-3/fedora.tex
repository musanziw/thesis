Fedora est une distribution Linux libre et open source, développée par la communauté et soutenue par Red Hat, une entreprise spécialisée dans les logiciels open source. Fedora est connue pour être une distribution Linux innovante et à la pointe de la technologie, avec une forte orientation vers les développeurs et les utilisateurs avancés.

Fedora utilise le gestionnaire de paquets RPM (RPM Package Manager) pour installer et gérer les logiciels, et propose une large sélection d'applications open source pour les tâches courantes, telles que la navigation sur le web, la création de documents, la gestion de fichiers, la programmation, etc. Il est également connu pour son support des technologies émergentes telles que Flatpak, qui permet d'emballer des applications pour qu'elles s'exécutent sur n'importe quelle distribution Linux.

Fedora est souvent utilisé comme une plateforme de développement pour les développeurs qui travaillent avec des technologies open source, telles que Python, Ruby, Java, Go, Rust, etc. Il est également populaire auprès des utilisateurs avancés qui cherchent à personnaliser leur environnement de bureau et à expérimenter avec de nouvelles technologies. Fedora est une distribution Linux communautaire et soutenue par la communauté, ce qui signifie que les utilisateurs peuvent participer au développement et à l'amélioration de la distribution en contribuant à la documentation, aux tests, aux correctifs, etc.