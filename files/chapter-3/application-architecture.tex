Nous avons choisi une architecture client-serveur pour notre application. Le
serveur est responsable de la logique métier et de la gestion des données, et
le client est responsable de l'interface utilisateur. Le client communique avec
le serveur via une API REST (Representational State Transfer), qui est un
ensemble de conventions et de bonnes pratiques pour la conception d'API web.
L'API REST permet au client d'effectuer des opérations CRUD (Create, Read,
Update, Delete) sur les données stockées sur le serveur. Nous avons choisi
d'utiliser une API REST, car elle est simple à mettre en œuvre et facile à
utiliser. Elle permet également de séparer la logique métier de l'interface
utilisateur, ce qui facilite la maintenance et l'évolutivité de l'application.
Nous avons choisi d'utiliser JSON (JavaScript Object Notation) comme format de
données pour l'API REST, car il est léger, facile à lire et à écrire, et
facilement extensible.