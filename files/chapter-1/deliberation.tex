La délibération est un processus de réflexion et de prise de décision collectif qui implique l'examen et l'évaluation de différentes options ou perspectives, souvent en vue d'arriver à un consensus ou à une décision commune. La délibération peut avoir lieu dans de nombreux contextes, tels que l'ensignement, les gouvernements, les organisations, les groupes de travail ou même les conversations informelles entre amis et collègues.

Dans une délibération, les participants discutent ouvertement et honnêtement des différents points de vue, arguments et preuves en faveur ou en défaveur de chaque option. L'objectif est souvent d'arriver à une décision qui soit juste, équitable et acceptable pour toutes les parties impliquées.

La délibération peut prendre différentes formes, allant des discussions informelles aux débats structurés et aux processus de consultation plus formels. Les participants peuvent également utiliser des outils tels que des sondages, des votes ou des simulations pour faciliter la prise de décision.

