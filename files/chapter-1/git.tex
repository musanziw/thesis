Git est un système de contrôle de version distribué utilisé pour le suivi des modifications apportées à un ensemble de fichiers, généralement du code source. Il est utilisé par les développeurs de logiciels pour collaborer sur des projets, suivre l'historique des modifications, effectuer des sauvegardes et des récupérations de code, et gérer les conflits de fusion.

Le fonctionnement de Git est basé sur un système de branches, où chaque développeur peut créer une branche de développement séparée pour travailler sur une fonctionnalité ou une correction de bogue spécifique. Les développeurs peuvent ensuite fusionner leurs branches avec la branche principale, généralement appelée "master", pour incorporer les modifications dans le code principal.

Git est largement utilisé dans l'industrie du développement de logiciels, ainsi que dans d'autres domaines où le suivi des modifications est important, tels que la rédaction de documents techniques, la gestion de projets et la collaboration sur des fichiers multimédias. Il est open source et gratuit, et est disponible pour Windows, macOS et Linux.