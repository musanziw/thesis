Node.js est une plateforme de développement open source qui permet d'exécuter du code JavaScript côté serveur. Elle a été développée en 2009 par Ryan Dahl et est basée sur le moteur JavaScript V8 de Google, qui est également utilisé dans le navigateur Chrome.

Node.js permet aux développeurs de créer des applications web dynamiques et évolutives en utilisant JavaScript des deux côtés de la plateforme, côté client et côté serveur. Il est souvent utilisé pour créer des serveurs web, des applications en temps réel, des API, des outils en ligne de commande, des scripts pour l'automatisation de tâches, etc.

Node.js est particulièrement adapté aux applications nécessitant une grande évolutivité, car il utilise un modèle événementiel non-bloquant (non-blocking event-driven model) qui permet de gérer de grandes quantités de connexions simultanées sans ralentir le serveur. Il est également facile à apprendre pour les développeurs qui connaissent déjà JavaScript, car il utilise la syntaxe et les structures de contrôle de base de JavaScript.

Node.js dispose d'une grande communauté de développeurs et de contributeurs qui ont créé de nombreuses bibliothèques et modules pour faciliter le développement d'applications. Il est également compatible avec de nombreuses autres technologies de développement web, telles que MongoDB, Express.js, React, Angular, etc.

En résumé, Node.js est une plateforme de développement JavaScript côté serveur qui permet de créer des applications web évolutives et dynamiques, en utilisant un modèle événementiel non-bloquant et en étant compatible avec de nombreuses autres technologies web.