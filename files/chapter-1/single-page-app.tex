Une SPA (Single Page Application) est une application web qui fonctionne sur une seule page, où tout le contenu est chargé dynamiquement sans avoir besoin de recharger la page complète lorsqu'un utilisateur interagit avec l'application.

Contrairement aux applications web traditionnelles qui ont plusieurs pages, une SPA est construite comme un ensemble de composants qui sont chargés dynamiquement, en réponse aux interactions de l'utilisateur. Les données sont généralement récupérées de manière asynchrone à l'aide d'API (Application Programming Interface) et de technologies côté client telles que JavaScript et AJAX.

Les avantages d'une SPA comprennent une expérience utilisateur plus rapide et plus fluide, car les interactions sont plus rapides et plus fluides que dans une application web traditionnelle. En outre, une SPA peut permettre aux développeurs de réduire la complexité du code et d'améliorer la maintenabilité en utilisant des frameworks et des bibliothèques qui facilitent le développement d'applications web dynamiques.

Cependant, une SPA peut également présenter des inconvénients, notamment une dépendance accrue aux technologies côté client, ce qui peut rendre l'application plus vulnérable aux attaques de sécurité et poser des problèmes de compatibilité avec certains navigateurs.