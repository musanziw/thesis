Le désir et la volonté d’offrir à la jeunesse une éducation de qualité sont les grandes motivations à l'origine de la création de l'Université Nouveaux Horizons. A cette immense volonté viennent en surcouche deux faits du vécu social qui sont : 
L'incitation et suggestion des tiers : 
Pendant plusieurs années des nombreuses personnes ont incités les promoteurs de l’Université Nouveaux Horizons à  créer une institution académique, dans le prolongement de l’enseignement de fin d’études  à l' GE D’OR, sans aucun doute on fait mention des parents des élèves de ladite école, des amis, et autres. 
Le regard de l’élève tourné vers l’étranger pour la suite des études après le cycle de l’enseignement secondaire : 
Le fait est que des nombreux élèves de la République Démocratique du Congo en fin de leurs études secondaires n’aspirent qu'à s'expatrier, pour aller s'inscrire, sous d’autres cieux car pensant que l’ailleurs et mieux que le chez soi; dans des Universités supposées de meilleure qualité. Par ceci on aperçoit aisément la critique souvent formulée à l’endroit des titres académiques décernés par l'Université en République Démocratique du Congo.
Par ailleurs, un autre fait ayant incité les initiateurs à créer l'Université Nouveaux Horizons est en quelque sorte ce qu’on pourrait appeler “l'héritage colonial” légué aux Universités Congolaises dans leur ensemble celui qui consiste à former des demandeurs d’emploi et non des créateurs d’emplois.

L'Université Nouveaux Horizons a pour but de dispenser des enseignements dans différentes filières, avec accent sur l'esprit d'entreprise (entrepreneuriat); de former des hommes et des femmes capables de participer efficacement au développement de la communauté, davantage comme des créateurs que des demandeurs d'emplois.
L'UNH assure un enseignement de qualité et de niveau international grâce à des compétences requises, judicieusement recrutées, pour cette fin.
