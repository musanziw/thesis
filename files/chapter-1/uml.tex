UML (Unified Modeling Language) est un langage de modélisation visuel utilisé pour représenter des systèmes logiciels. UML fournit un ensemble de notations graphiques standardisées pour décrire les différents aspects d'un système, tels que les exigences, la conception, la mise en œuvre et le déploiement.

Voici les principales notations graphiques utilisées dans UML :
\subsection{Diagramme de classes}\label{subsec:diagrammes-de-classe}
 Utilisé pour représenter la structure statique du système, en définissant les classes, les attributs, les méthodes et les relations entre les classes.
\subsection{Diagramme de cas d'utilisation}\label{subsec:diagrammes-de-cas-utilisation}
Utilisé pour représenter les interactions entre les acteurs et le système, et pour définir les exigences fonctionnelles du système.
\subsection{Diagramme d'objets}\label{subsec:diagrammes-d-objets}
Utilisé pour représenter les instances des classes et les relations entre les objets.
\subsection{Diagramme de séquence}\label{subsec:diagrammes-de-sequence}
Utilisé pour représenter les interactions entre les objets dans une séquence temporelle, en montrant les messages échangés entre les objets.
\subsection{Diagramme de collaboration}\label{subsec:diagrammes-de-collaboration}
Diagramme de collaboration : Utilisé pour représenter les interactions entre les objets en termes de messages échangés et de relations entre les objets.
\subsection{Diagramme d'états-transitions}\label{subsec:diagrammes-d-etat-transition}
Utilisé pour représenter le comportement dynamique d'un objet ou d'un système, en montrant les états possibles et les transitions entre ces états.
\subsection{Diagramme d'activités}\label{subsec:diagrammes-d-activites}
Utilisé pour représenter le flux de contrôle dans un processus ou une méthode, en montrant les activités, les décisions et les boucles.
\subsection{Diagramme de déploiement}\label{subsec:diagrammes-de-deploiement}
Utilisé pour représenter la configuration physique du système, en montrant les composants matériels et logiciels et leurs relations.

