Un langage de programmation est un langage formel utilisé pour écrire des instructions qui seront exécutées par un ordinateur. Les langages de programmation sont utilisés pour créer des logiciels, des applications, des sites web, des systèmes d'exploitation, des jeux vidéo et de nombreux autres types de programmes informatiques.

Un langage de programmation se compose d'un ensemble de règles syntaxiques et sémantiques qui définissent la structure et le sens des instructions. Les instructions sont écrites sous forme de code source, qui est ensuite traduit en code machine par un compilateur ou un interprète.

Les langages de programmation peuvent être classés en plusieurs catégories, telles que :
\begin{enumerate}
    \item Les langages de programmation impératifs : qui décrivent les étapes à suivre pour accomplir une tâche. Les langages de programmation impératifs peuvent être classés en deux catégories : les langages de programmation procéduraux et les langages de programmation orientés objet.
    \item Les langages de programmation fonctionnels : qui se concentrent sur les fonctions mathématiques et l'évaluation d'expressions, etc. 
    \item Les langages de programmation orientés objet : qui sont basés sur des objets et leurs interactions,  etc. Les langages de programmation orientés objet peuvent être classés en deux catégories : les langages de programmation orientés objet basés sur les classes et les langages de programmation orientés objet basés sur les prototypes.
    \item Les langages de script : qui sont utilisés pour automatiser des tâches et des processus, etc. Les langages de script sont généralement interprétés plutôt que compilés.
    \item Les langages de programmation déclaratifs : qui décrivent le résultat souhaité plutôt que les étapes à suivre pour l'obtenir, 
\end{enumerate}

Il existe de nombreux langages de programmation différents, chacun ayant ses propres avantages et inconvénients. Voici quelques-uns des langages de programmation les plus courants :
\begin{itemize}
    \item C++,
    \item JavaScript,
    \item Java,
    \item CSharp,
    \item Ruby,
    \item Python,
    \item Etc.
\end{itemize}

