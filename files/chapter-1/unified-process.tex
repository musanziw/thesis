Le Processus Unifié (PU), également connu sous le nom de Rational Unified Process (RUP), est une méthode de développement logiciel qui suit une approche itérative et incrémentale. Le PU est un cadre de développement logiciel qui fournit des directives détaillées pour toutes les étapes du développement logiciel, de la planification initiale à la maintenance et la mise à jour continue du logiciel.
Le processus Unifié est une méthode de développement logiciel agile de la lignée dite unifiée
qui nécessite une adaptation à chaque projet qui y a recours, elle est génerique, itérative
et incrémentale.

Le PU est basé sur quatre principes fondamentaux :
\begin{itemize}
    \item La modélisation visuelle : Le PU utilise des diagrammes UML (Unified Modeling Language) pour représenter le système logiciel.
    \item La gestion de projet : Le PU utilise une approche de gestion de projet itérative et agile qui permet de s'adapter aux changements des exigences et aux problèmes qui surgissent pendant le développement logiciel.
    \item La vérification et la validation : Le PU insiste sur la vérification et la validation du logiciel à chaque étape du processus pour s'assurer que le logiciel est conforme aux exigences du client et qu'il est de haute qualité.
    \item L'architecture centrée sur les cas d'utilisation : Le PU se concentre sur la conception de l'architecture du logiciel en se basant sur les cas d'utilisation qui décrivent les interactions entre les utilisateurs et le système.
\end{itemize}

Le PU est structuré en quatre phases principales :
\begin{enumerate}
    \item La phase d'inception : Cette phase consiste à définir le périmètre du projet, à identifier les objectifs et les exigences, et à élaborer une vision claire du système logiciel.
    \item La phase d'élaboration : Cette phase consiste à élaborer une architecture détaillée du système logiciel et à développer un plan de développement détaillé.
    \item La phase de construction : Cette phase consiste à développer et à tester le logiciel en utilisant une approche itérative et incrémentale.
    \item La phase de transition : Cette phase consiste à déployer le logiciel et à assurer la formation des utilisateurs, la maintenance et la mise à jour continue.
\end{enumerate}