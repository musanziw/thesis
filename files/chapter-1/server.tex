Dans le contexte des applications web, un serveur est un ordinateur ou un logiciel qui fournit des services à d'autres ordinateurs ou logiciels. Le serveur est souvent utilisé pour héberger des applications web, des sites web ou des API, et pour répondre aux demandes des clients, qui peuvent être des navigateurs web ou des applications clientes.

Dans le contexte de l'architecture client-serveur, le serveur est la partie centrale qui gère les données et les traitements, tandis que les clients communiquent avec le serveur pour obtenir des informations et effectuer des opérations. Le serveur peut être un serveur de base de données, un serveur d'application, un serveur de fichiers, etc.

Dans le contexte des applications distribuées, un serveur peut être un nœud qui fournit des services à d'autres nœuds dans un réseau peer-to-peer ou dans un réseau de type client-serveur.

\subsection{Le serveur mutualisé}\label{subsec:serveur-mutualise}
Un serveur mutualisé est un type de serveur qui est partagé entre plusieurs utilisateurs, qui utilisent tous les mêmes ressources du serveur, telles que la mémoire, la puissance de traitement, l'espace de stockage, etc.

Dans un environnement de serveur mutualisé, chaque utilisateur dispose d'un compte d'hébergement séparé, mais tous les comptes partagent les mêmes ressources du serveur. Cela signifie que les performances et la stabilité du serveur peuvent être affectées par les besoins en ressources des autres utilisateurs, en particulier si certains utilisateurs consomment une grande quantité de ressources.

Les serveurs mutualisés sont souvent utilisés pour héberger des sites web de petite ou moyenne taille, des blogs, des forums ou des applications qui n'ont pas besoin de beaucoup de ressources système. Ils sont également moins coûteux que les autres options d'hébergement, comme les serveurs dédiés ou les VPS.

Cependant, en raison du partage des ressources, les serveurs mutualisés peuvent présenter des inconvénients, tels que des temps de chargement plus lents, des problèmes de sécurité et des limitations sur les logiciels et les fonctionnalités disponibles.

\subsection{Le serveur dédié}\label{subsec:serveur-dedie}
Un VPS (Virtual Private Server) est un type de serveur qui permet à plusieurs utilisateurs de partager les ressources d'un même serveur physique, tout en ayant chacun leur propre environnement de serveur virtuel isolé.
Un VPS (Virtual Private Server) est un type de serveur qui permet à plusieurs utilisateurs de partager les ressources d'un même serveur physique, tout en ayant chacun leur propre environnement de serveur virtuel isolé.

En d'autres termes, un VPS est un serveur virtuel qui fonctionne sur un serveur physique et qui est divisé en plusieurs instances isolées les unes des autres. Chaque instance, appelée "machine virtuelle", dispose de son propre système d'exploitation, de ses propres ressources système et de ses propres applications, ce qui permet à chaque utilisateur de personnaliser son environnement de serveur selon ses besoins.

Les VPS sont souvent utilisés pour héberger des sites web, des applications et des services en ligne, car ils offrent une flexibilité et une évolutivité importantes. Ils permettent également aux utilisateurs de bénéficier des avantages d'un serveur dédié, tels que des performances élevées, une sécurité renforcée et un accès root complet, sans avoir à investir dans leur propre matériel ou à supporter les coûts associés à la gestion d'un serveur physique.
En d'autres termes, un VPS est un serveur virtuel qui fonctionne sur un serveur physique et qui est divisé en plusieurs instances isolées les unes des autres. Chaque instance, appelée "machine virtuelle", dispose de son propre système d'exploitation, de ses propres ressources système et de ses propres applications, ce qui permet à chaque utilisateur de personnaliser son environnement de serveur selon ses besoins.

Les VPS sont souvent utilisés pour héberger des sites web, des applications et des services en ligne, car ils offrent une flexibilité et une évolutivité importantes. Ils permettent également aux utilisateurs de bénéficier des avantages d'un serveur dédié, tels que des performances élevées, une sécurité renforcée et un accès root complet, sans avoir à investir dans leur propre matériel ou à supporter les coûts associés à la gestion d'un serveur physique.