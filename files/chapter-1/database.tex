Une base de données est une collection organisée de données qui sont stockées de manière à permettre un accès facile et efficace, ainsi que des opérations de gestion de données. Elle est utilisée pour stocker et organiser des informations pour une utilisation ultérieure.

Dans le contexte du développement logiciel, une base de données est souvent utilisée pour stocker les données de l'application, telles que les informations utilisateur, les commandes, les messages, les transactions, etc. Elle permet également aux développeurs d'accéder facilement à ces données, de les organiser et de les manipuler à l'aide de requêtes SQL (Structured Query Language), qui est le langage de programmation standard pour interagir avec les bases de données relationnelles.

\subsection{Bases de données relationnelles}\label{subsec:relational-databases}
Les bases de données relationnelles sont les plus couramment utilisées dans le développement logiciel. Elles stockent les données dans des tables avec des relations entre elles, et permettent aux développeurs de les manipuler à l'aide de requêtes SQL. MySQL, PostgreSQL et Microsoft SQL Server sont des exemples de bases de données relationnelles populaires.

\subsection{Bases de données NoSQL}\label{subsec:nosql-databases}
Les bases de données NoSQL (Not Only SQL) sont une alternative aux bases de données relationnelles. Elles sont souvent utilisées pour des applications à grande échelle et nécessitant une grande évolutivité. Les bases de données NoSQL stockent les données sous forme de documents, de graphes ou de paires clé-valeur, et utilisent souvent des langages de requête spécifiques plutôt que SQL. MongoDB, Cassandra et Couchbase sont des exemples de bases de données NoSQL populaires.

\subsection{Bases de données orientées objet}\label{subsec:object-oriented-databases}
Les bases de données orientées objet sont conçues pour stocker des objets plutôt que des tables. Elles sont souvent utilisées dans les applications de programmation orientée objet, où les données sont stockées sous forme d'objets avec des attributs et des méthodes. db4o et ObjectDB sont des exemples de bases de données orientées objet populaires.

\subsection{Bases de données en mémoire}\label{subsec:in-memory-databases}
Les bases de données en mémoire stockent les données en mémoire vive plutôt que sur un disque dur, ce qui permet des temps d'accès plus rapides et une meilleure performance. Elles sont souvent utilisées dans les applications nécessitant une réponse rapide, comme les jeux en ligne et les applications financières. Redis et Memcached sont des exemples de bases de données en mémoire populaires.