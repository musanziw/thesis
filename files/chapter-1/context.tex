\begingroup
\paragraph{}
 Ce présent travail s'articule autour de la délibération des étudiants de l'Université Nouveaux Horizons, il est important de retenir par ailleurs que la délibération se fait comme suit :
 \begin{itemize}
 	\item[\labelitemii] Après les examens, pour tous les cours concernés; les professeurs envoient les moyennes annuelles pour certains; d’autres simplement les cotes des travaux au format Excel, Pdf ou en manuscrit au secrétariat général académique en copie à la faculté concernée qui a l'occurrence est représentée par un doyen,
 	\item[\labelitemii] La faculté (le doyen) transmet ensuite les cotes reçues dans l’un des formats évoqués ci-haut aux membres du jury.,
 	\item[\labelitemii] Ce dernier après réception des côtes peu importe le format, les encodent sous Excel et procède à la délibération après le calcul des moyennes annuelles premier semestre; En ce qui concerne la fin de l'année après réception de toutes les côtes, le calcul du pourcentage de chaque étudiant se fait en gardant le regard fixé sur les critères de délibération entre autre le crédit du cours car étant dans le système LMD ensuite vient la délibération des étudiants,
 	\item[\labelitemii] Après ce long processus, l'heure vient de communiquer aux étudiants leurs sort soit par mail, soit en présentiels ils viennent récupérer leurs relevés des cotes.
 \end{itemize} 

Au passage nous allons signaler l'intervention des quelques acteurs principaux dont :
\begin{itemize}
	\item[\labelitemii] Les professeurs,
	\item[\labelitemii] Le secrétaire général académique,
	\item[\labelitemii] Le doyen,
	\item[\labelitemii] Le jury.
\end{itemize}
\endgroup
