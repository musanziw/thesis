Le déploiement en développement logiciel fait référence au processus de mise en production d'une application ou d'un logiciel, c'est-à-dire de le rendre disponible et fonctionnel pour les utilisateurs finaux.

Le déploiement implique plusieurs étapes, notamment la préparation de l'environnement de production, la configuration des serveurs, la mise en place des bases de données et des systèmes de stockage, la compilation du code source, la mise en place des fichiers de configuration, la vérification des dépendances, et la résolution de tout problème de compatibilité.

Une fois que l'application est prête à être déployée, elle est souvent transférée sur un serveur de production ou un environnement de cloud computing, où elle peut être mise à la disposition des utilisateurs finaux. Les développeurs peuvent mettre en place des stratégies de déploiement automatique pour faciliter le processus de déploiement et réduire le risque d'erreurs humaines.

Le déploiement est une étape clé du cycle de vie du développement logiciel, car il permet de rendre l'application disponible pour les utilisateurs finaux, ce qui peut générer des retours d'utilisateurs et des données d'utilisation qui peuvent être utilisés pour améliorer l'application dans les versions futures.