L'Université Nouveaux Horizons est une institution moderne qui s'efforce constamment d'améliorer ses processus, notamment en ce qui concerne la délibération des étudiants. Dans ce mémoire, nous présentons une solution informatique sous forme d'application pour aider l'université dans ce processus.

La délibération des étudiants est un processus complexe qui implique l'analyse de plusieurs critères, notamment leur performance académique. L'application que nous proposons prend en compte cet aspect en permettant de suivre les données des étudiants, telles que leurs cours en complément et leurs relevés de notes. Elle permet également de calculer les pourcentages et de générer une grille de délibération ainsi que des relevés de notes personnalisés pour chaque étudiant.

L'application facilite la prise de décision du jury en fournissant des informations pertinentes et en simplifiant le processus de délibération. Elle permet également à l'université de gagner en temps et en efficacité dans la gestion des données des étudiants.

Notre solution est le fruit d'une analyse minutieuse des pratiques existantes et des standards en vigueur dans d'autres universités. Elle répond ainsi de manière adéquate aux besoins spécifiques de l'Université Nouveaux Horizons.

Nous avons conçu notre application comme une solution extensible, grâce à une API Rest qui permet une communication aisée avec d'autres services tiers tels que Moodle ou Google Classroom. Cette flexibilité assure une certaine indépendance et une évolutivité de la solution.

En somme, notre application offre une solution efficace et adaptée pour la délibération des étudiants, qui permettra à l'Université Nouveaux Horizons de moderniser ses processus et de gagner en efficacité tout en assurant un suivi personnalisé des étudiants.