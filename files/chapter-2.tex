Avant de commencer à développer notre application, nous avons d'abord
comparer les différentes applications existantes pour en dégager les
fonctionnalités clés, nous avons ensuite mis sur un balance les avantages
et les inconvénients de chacune d'elles pour enfin proposer une solution qui
répondrait aux besoins de l'Université Nouveaux Horizons.

\section{Présentation des principales applications existantes}\label{sec:definitions-et-concepts-cles}
Nous sommes conscients qu'il en existe certainement
lusieurs autres, mais nous avons choisi de nous limiter à celles mentioner ci dessous
car elles sont les plus utilisées, les plus populaires et couvrent la plupart des besoins.

\subsection{Moodle}\label{subsec:moodle}
\begin{enumerate}
    \item \textbf{Présentation de l'application} \newline Moodle est un système de gestion de l'apprentissage (LMS) open source, c'est-à-dire un logiciel permettant la création, la gestion, la distribution et la surveillance de cours en ligne. Il permet aux enseignants et aux formateurs de créer des cours en ligne interactifs, d'organiser des activités d'apprentissage, de communiquer avec les étudiants et de suivre leur progression.
    \item \textbf{Fonctionalités clés}
        \begin{itemize}
            \item Gestion des cours : Moodle permet de créer des cours en ligne et d'organiser des activités d'apprentissage en utilisant des outils tels que des forums de discussion, des leçons, des wikis, des glossaires, des quiz, des devoirs, des sondages, etc.
            \item Gestion des utilisateurs : Moodle permet de gérer les utilisateurs, de les inscrire aux cours, de les suivre et de leur attribuer des rôles spécifiques (enseignant, étudiant, administrateur, etc.).
            \item Communication : Moodle offre plusieurs outils de communication, y compris les forums de discussion, les messages privés, les chats en temps réel et les webinaires.
            \item Suivi des progrès : Moodle permet aux enseignants de suivre les progrès des étudiants en visualisant leurs notes, leurs activités et leurs contributions sur les forums de discussion et autres outils.
            \item Accessibilité : Moodle est conforme aux normes d'accessibilité pour les personnes handicapées, ce qui permet à tous les utilisateurs de profiter pleinement des activités d'apprentissage en ligne.
        \end{itemize}
    \item \textbf{Inconvénients et limites}
        \begin{itemize}
            \item Expertise technique requise : Moodle est un logiciel complexe dont l'installation et la configuration requièrent une certaine expertise technique.
            \item Sécurité : Moodle est une cible populaire pour les pirates informatiques, il est donc important de prendre des mesures pour sécuriser votre site Moodle. Cela inclut l'utilisation de mots de passe forts, l'installation de mises à jour de sécurité, et la surveillance du site pour toute activité suspecte.
            \item Evolutivité : Moodle peut être adapté à un grand nombre d'utilisateurs, mais il peut être difficile de gérer un grand site Moodle.
            \item Moodle est conçu pour répondre aux besoins généraux de l'enseignement en ligne. Il n'est pas conçu pour répondre aux besoins spécifiques d'une institution.
        \end{itemize}
\end{enumerate}

\subsection{PowerSchool}\label{subsec:powerschool}
\begin{enumerate}
    \item \textbf{Présentation de l'application} \newline PowerSchool est un système d'information pour les établissements scolaires qui permet de gérer les informations relatives aux étudiants, aux enseignants et aux programmes académiques. Il est utilisé par plus de 45 millions d'utilisateurs dans le monde entier, notamment dans les écoles primaires, secondaires et les établissements d'enseignement supérieur.
    \item \textbf{Fonctionalités clés}
        \begin{itemize}
            \item Gestion des inscriptions et des admissions : PowerSchool permet de gérer les inscriptions et les admissions des étudiants, y compris la collecte des informations de base, la gestion des documents requis, la communication avec les parents et les étudiants, et la planification des cours.
            \item Gestion des notes et des absences : PowerSchool permet aux enseignants d'enregistrer les notes et les absences des étudiants, et de partager les informations avec les parents et les étudiants.
            \item Gestion des emplois du temps et des calendriers académiques : PowerSchool permet de gérer les emplois du temps des enseignants et des étudiants, ainsi que les calendriers académiques et les événements scolaires.
            \item Gestion des frais de scolarité et des paiements : PowerSchool permet de gérer les frais de scolarité et les paiements, y compris la collecte et le suivi des paiements, la génération de factures, et la communication avec les parents et les étudiants.
            \item Communication avec les parents et les étudiants : PowerSchool permet de communiquer avec les parents et les étudiants via des messages personnalisés, des bulletins électroniques, des alertes automatiques, et d'autres outils de communication.
            \item Gestion des bibliothèques : PowerSchool permet de gérer les bibliothèques de l'établissement, y compris la gestion des prêts de livres, la gestion des retours, et la gestion des réservations de livres.
            \item Rapports et analyses : PowerSchool permet de générer des rapports et des analyses sur les performances académiques des étudiants, les tendances de fréquentation, les statistiques de paiement, et bien plus encore.
        \end{itemize}
    \newpage
    \item \textbf{Inconvénients et limites}
        \begin{itemize}
            \item Coût : Le coût initial de mise en place de PowerSchool peut être élevé, en particulier pour les établissements scolaires plus petits. Les coûts peuvent également augmenter avec l'ajout de fonctionnalités supplémentaires.
            \item Complexité : PowerSchool est un système complexe qui peut nécessiter une formation pour les administrateurs, les enseignants et les utilisateurs finaux. Les établissements scolaires doivent également consacrer du temps et des ressources à la configuration et à la gestion du système.
            \item Personnalisation : Bien que PowerSchool offre de nombreuses fonctionnalités, il peut être difficile de personnaliser le système pour répondre aux besoins spécifiques de chaque établissement. Les établissements peuvent être limités dans leur capacité à personnaliser le système en fonction de leurs besoins spécifiques.
            \item Dépendance : Les établissements qui utilisent PowerSchool dépendent du système pour gérer les données des étudiants et de l'établissement. En cas de panne du système ou de perte de données, cela peut avoir des conséquences importantes pour l'établissement.
        \end{itemize}
\end{enumerate}

Apres la présentation de ces deux applications, nous nous intéréssé à la manière dont les
universités soeurs de l'Université Nouveaux Horizons gèrent leurs données et leurs
processus académiques, l'une d'elle s'est démarquée par son approche innovante.

\subsection{Esis Salama}\label{subsec:esis-salamae}
\begin{enumerate}
    \item \textbf{Présentation} \newline Esis salama est une institution d'enseignement supérieure, ce qui lui vaut une mention dans ce présent travail c'est la manière dont elle gère la délibération.
    \item \textbf{Outils utilisés}
        \begin{itemize}
            \item Excel : Ce dernier est un tableur édité par Microsoft pour les systèmes d'exploitation Windows et Mac OS X. Il est principalement utilisé pour le calcul, l'analyse de données et les graphiques. Excel est l'un des programmes les plus populaires pour la gestion des données et des calculs.
            \item Plateforme développée en interne :  Cette dernière facilite la publication des résultats des étudiants et permet aux étudiants de consulter leurs résultats et de faire des recours en ligne.
        \end{itemize}
    \item \textbf{Inconvénients et limites}
        \begin{itemize}
            \item Cette façon de faire est très chronophage,
            \item Nécessite beaucoup de ressources humaines
        \end{itemize}
\end{enumerate}

\section{conclusion}\label{sec:conclusion}
En conclusion nous retiendront que l'idéal serait de produire
une application qui comblerait les lacunes des applications existantes
et les étendrait mais l'idéal n'est pas toujours réalisable.
Nous proposons donc une solution qui resoud les gros inconvenients
et limitations des applications existantes et qui est facilement
extensible et modifiable.

Les apports sont :
\begin{itemize}
    \item Le fait que cette application soit développée en interne permettra de l'adapter aux besoins de l'Université Nouveaux Horizons et de rester indépendant des éditeurs de logiciels.
    \item L'application sera facilement extensible et modifiable.
    \item L'application peut facilement s'intégrer au système existant.
    \item L'application permettra de réduire les coûts en temps et ressources humaines car elle fera la grande partie du travail certes sous la supervision humaine.
\end{itemize}


