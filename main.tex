\documentclass[12pt, a4paper]{report}
\usepackage[a4paper, margin=2cm]{geometry}
\usepackage{graphicx}
\usepackage[T1]{fontenc}
\usepackage{natbib}
\usepackage[french]{babel}
\usepackage{fancyhdr}
\usepackage{dirtytalk}
\usepackage{subfiles}
\usepackage{mathptmx}
\graphicspath{{images/}}
\begin{document}
    \subfile{subfiles/cover-page}

    \addcontentsline{toc}{chapter}{Épigraphe}
    \chapter*{Épigraphe}\label{cha:epigraphe}

    \addcontentsline{toc}{chapter}{Dédicace}
    \chapter*{Dédicace}\label{cha:dedicace}

    \addcontentsline{toc}{chapter}{Remerciements}
    \chapter*{Remerciements}\label{cha:remerciements}

    \addcontentsline{toc}{chapter}{Introduction générale}
    \chapter*{Introduction générale}\label{cha:introduction-generale}
    \subfile{subfiles/introduction.tex}

    \tableofcontents

    \chapter{Étude préalable}\label{cha:etude-prealable}  

    \subsection{Étude de l'existant}\label{subsec:etude-de-l-existant}
    \subfile{subfiles/existing}
  
    \newpage
    \subsection{Présentation de l’entreprise}\label{subsec:presentation-de-l-entreprise}
    \subfile{subfiles/entreprise}
    
    
    \subsection{Objectifs et vision de l’entreprise}\label{subsec:objectifs-et-vision-de-l-entreprise}
    \subfile{subfiles/goal-vision}
    
   	\subsection{Solution informatique}\label{subsec:solution-informatique}
    \paragraph{}
    La solution proposée est un logiciel dit applicatif, spécifique et propriétaire
    car étant en rapport avec une tâche
    particulière ici qui est la délibération des étudiants, 
    étant destinée à résoudre la dite tâche au sein de l'UNH\footnote{Université Nouveaux Horizons} et
    tous les droits sont réservés.

   	\chapter{Analyse}\label{cha:analyse}
    \subsection{Introduction}\label{subsec:introduction}
    \subsection{Diagramme des classes}\label{subsec:diagramme-des-classes-de-l-application}
    \subsection{Diagramme d'objets}\label{subsec:diagramme-des-objets-de-l-application}
    \subsection{Diagramme de cas d’utilisation}\label{subsec:diagramme-de-cas-d-utilisation}
   	\subsection{Diagramme de séquence}\label{subsec:diagramme-de-sequence}
    \subsection{Diagramme d’activités}\label{subsec:diagramme-d-activites}
    
    \chapter{Conception}\label{sec:conception}
    \subsection{Conception du système}\label{subsec:conception-du-systeme}
    \subsection{Architecture du système}\label{subsec:architecture-du-systeme}
    \subsection{Mise au point du plan de réutilisation}\label{subsec:mise-au-point-du-plan-de-reutilisation}
    \subsection{Conception des classes système}\label{subsec:conception-des-classes-systeme}
    
    \chapter{Implémentation}\label{sec:implementation}
    \subsection{Introduction}\label{subsec:introduction-implementation}
   	\subsection{Choix des outils}\label{subsec:choix-des-outils}
   	\subsection{Présentation des résultats}\label{subsec:presentation-des-resultats}
   	
    \chapter{Conclusion}\label{sec:conclusion}
    \chapter{Bibliographie}\label{sec:bibliographie}
\end{document}
