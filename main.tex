\documentclass[12pt, a4paper]{report}
\usepackage[a4paper, margin=2cm]{geometry}
\usepackage{graphicx}
\usepackage[T1]{fontenc}
\usepackage{natbib}
\usepackage[french]{babel}
\usepackage{fancyhdr}
\usepackage{dirtytalk}
\usepackage{subfiles}
\usepackage{mathptmx}
\graphicspath{{images/}}
\begin{document}
    \subfile{files/cover/cover-page}

    \addcontentsline{toc}{chapter}{Résumé}
    \chapter*{\uppercase{\large{\textmd{Résumé}} \vspace{0.6cm} \hrule}}\label{cha:resume}

    \addcontentsline{toc}{chapter}{Abstract}
    \chapter*{\uppercase{\large{\textmd{Abstract}} \vspace{0.6cm} \hrule}}\label{cha:abstract}

    \addcontentsline{toc}{chapter}{Rémerciements}
    \chapter*{\uppercase{\large{\textmd{Rémerciements}} \vspace{0.6cm} \hrule}}\label{cha:remerciements}
    

    \renewcommand*\contentsname{\uppercase{\large{\textmd{Table des matières}}} \vspace{0.5cm} \hrule }
    \tableofcontents

    \addcontentsline{toc}{chapter}{Introduction générale}
    \chapter*{Introduction générale}\label{cha:introduction-generale}
    \subfile{files/introduction/general-introduction}

    \section{Contexte et motivation}\label{subsec:contexte-et-motivation}
    \subfile{files/introduction/context-and-motivation.tex}

    \section{Problématique}\label{sec:problematique}
    \subfile{files/introduction/problematic.tex}

    \section{Etat de la question}\label{sec:etat-de-l-art}
    \section{Objectifs}\label{sec:objectifs}
    \section{Méthodes et techniques}\label{sec:methode}


    \chapter{État de l'art}\label{cha:etat-de-l-art}
    \section{Présentation des principales applications existantes}\label{sec:definitions-et-concepts-cles}
    \subsection{Types d'applications existantes}\label{subsec:types-d-applications-d-aide-a-la-deliberation}
    \subsection{Avantages et inconvénients des applications existantes}\label{subsec:avantages-et-inconvenients-des-applications-existantes}
    \subsection{Évaluation des fonctionnalités clés}\label{subsec:evaluation-des-fonctionnalites-cles}
    \subsection{Comparaison des différentes applications}\label{subsec:comparaison-des-differentes-applications}
    \subsection{Analyse des besoins de l'Université Nouveaux Horizons}\label{subsec:analyse-des-besoins-des-etudiants-de-l-universite-nouveaux-horizons}
    \subsection{Identification des fonctionnalités clés pour répondre à ces besoins}\label{subsec:identification-des-fonctionnalites-cles-pour-repondre-a-ces-besoins}
    \subsection{Justification des choix de fonctionnalités}\label{subsec:justification-des-choix-de-fonctionnalites-pour-votre-application}
    \section{conclusion}\label{sec:conclusion}
    \subsection{Synthèse des résultats de l'état de l'art}\label{subsec:synthese-des-resultats-de-l-etat-de-l-art}
    \subsection{Limites et perspectives de l'état de l'art}\label{subsec:limites-et-perspectives-de-l-etat-de-l-art}
    \subsection{Implications pour le développement de votre application}\label{subsec:implications-pour-le-developpement-de-votre-application}
    
    \chapter{Conception de l'application}\label{cha:conception-de-l-application}
    \section{Spécification des besoins fonctionnels et non fonctionnels}\label{sec:specification-des-besoins-fonctionnels-et-non-fonctionnels}
    \section{Architecture logicielle et choix technologiques}\label{sec:architecture-logicielle-et-choix-technologiques}
    \section{Conception de l'interface utilisateur}\label{sec:conception-de-l-interface-utilisateur}

    \chapter{Réalisation de l'application}\label{cha:realisation-de-l-application}
    \section{Élaboration des modules fonctionnels}\label{sec:elaboration-des-modules-fonctionnels}
    \section{Intégration des modules}\label{sec:integration-des-modules}
    \section{Tests et validation}\label{sec:tests-et-validation}

    \chapter{Évaluation de l'application}\label{cha:evaluation-de-l-application}
    \section{Évaluation de l'utilisabilité}\label{sec:evaluation-de-l-utilisabilite}
    \section{Évaluation de l'efficacité de l'application}\label{sec:evaluation-de-l-efficacite-de-l-application}
    \section{Propositions d'amélioration}\label{sec:propositions-d-amelioration}

    \addcontentsline{toc}{chapter}{Conclusion générale}
    \chapter*{Conclusion générale}\label{cha:conclusion-generale}
    \section{Bilan du projet}\label{sec:bilan-du-projet}
    \section{Apports et perspectives}\label{sec:apports-et-perspectives}
    \section{Limites et recommandations}\label{sec:limites-et-recommandations}
    
\end{document}
