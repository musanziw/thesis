\documentclass[12pt, a4paper]{report}
\usepackage[a4paper, margin=2cm]{geometry}
\usepackage{graphicx}
\usepackage[T1]{fontenc}
\usepackage{natbib}
\usepackage[french]{babel}
\usepackage{fancyhdr}
\usepackage{dirtytalk}
\usepackage{subfiles}
\usepackage{mathptmx}
\usepackage{enumitem}
\graphicspath{{images/}}
\begin{document}
    \subfile{files/cover/cover-page}

    \addcontentsline{toc}{chapter}{Résumé}
    \chapter*{\uppercase{\large{\textmd{Résumé}} \vspace{0.6cm} \hrule}}\label{cha:resume}

    \addcontentsline{toc}{chapter}{Abstract}
    \chapter*{\uppercase{\large{\textmd{Abstract}} \vspace{0.6cm} \hrule}}\label{cha:abstract}

    \addcontentsline{toc}{chapter}{Rémerciements}
    \chapter*{\uppercase{\large{\textmd{Rémerciements}} \vspace{0.6cm} \hrule}}\label{cha:remerciements}

    \renewcommand*\contentsname{\uppercase{\large{\textmd{Table des matières}}} \vspace{0.5cm} \hrule }
    \tableofcontents

    \renewcommand*\listfigurename{\uppercase{\large{\textmd{Table des figures}}} \vspace{0.5cm} \hrule }
    \listoffigures

    \addcontentsline{toc}{chapter}{Introduction générale}
    \chapter*{Introduction générale}\label{cha:introduction-generale}
    \subfile{files/introduction/general-introduction}

    \section{Contexte et motivation}\label{subsec:contexte-et-motivation}
    \subfile{files/introduction/context-and-motivation.tex}

    \section{Problématique}\label{sec:problematique}
    \subfile{files/introduction/problematic.tex}

    \section{Etat de la question}\label{sec:etat-de-l-art}
    \subfile{files/introduction/state-of-question.tex}

    \section{Objectifs}\label{sec:objectifs}
    \subfile{files/introduction/goals.tex}

    \section{Méthodes et techniques}\label{sec:methode}
    \subfile{files/introduction/methods-and-techniques.tex}

    \chapter{Généralités}\label{cha:generalites}

    \section{La délibération}\label{subsec:deliberation}
    \subfile{files/chapter-1/deliberation.tex}

    \section{Une application}\label{subsec:application}
    \subfile{files/chapter-1/application.tex}

    \section{Un étudiant}\label{subsec:etudiants}
    \subfile{files/chapter-1/student.tex}

    \section{Web}\label{subsec:web}
    \subfile{files/chapter-1/web.tex}

    \section{Une API}\label{subsec:api}
    \subfile{files/chapter-1/api.tex}

    \section{L'université Nouveaux Horizons}\label{subsec:unh}
    \subfile{files/chapter-1/unh.tex}

    \section{Les méthodes de développement logiciel}\label{subsec:methode-de-developpement-logiciel}
    \subfile{files/chapter-1/software-dev-methods.tex}

    \subsection{Le processus unifié}\label{subsec:processus-unifie}
    \subfile{files/chapter-1/unified-process.tex}

    \section{Git}\label{sec:git-1}
    \subfile{files/chapter-1/git.tex}

    \section{GitHub}\label{sec:github-1}
    \subfile{files/chapter-1/github.tex}

    \section{Linux}\label{sec:linux}
    \subfile{files/chapter-1/linux.tex}
   
    \section{La modélisation}\label{sec:modelisation}
    \subfile{files/chapter-1/modeling.tex}

    \section{UML (Unified Modeling Language)}\label{sec:uml}
    \subfile{files/chapter-1/uml.tex}

    \section{HTML (HyperText Markup Language)}\label{sec:html}
    \subfile{files/chapter-1/html.tex}

    \section{CSS (Cascading Style Sheets)}\label{sec:css}
    \subfile{files/chapter-1/css.tex}

    \section{Un language de programmation}\label{sec:language-de-programmation}
    \subfile{files/chapter-1/programming.tex}

    \section{Un framework}\label{sec:framework}
    \subfile{files/chapter-1/framework.tex}

    \section{NodeJS}\label{sec:nodejs}
    \subfile{files/chapter-1/nodejs.tex}

    \section{Une base des données}\label{sec:base-de-donnees}
    \subfile{files/chapter-1/database.tex}

    \section{Une SPA (Single Page Application)}\label{sec:spa}
    \subfile{files/chapter-1/single-page-app.tex}
    
    \section{UI/UX}\label{sec:ui-ux}
    \subfile{files/chapter-1/user-interface.tex}
    \subfile{files/chapter-1/user-experience.tex}

    \chapter{État de l'art}\label{cha:etat-de-l-art}
    \subfile{files/chapter-2/state-of-art.tex}

    \section{Présentation des principales applications existantes}\label{sec:definitions-et-concepts-cles}
    \subfile{files/chapter-2/presentation-of-main-existing-applications.tex}

    \subsection{Moodle}\label{subsec:moodle}
    \subfile{files/chapter-2/moodle.tex}

    \subsection{PowerSchool}\label{subsec:powerschool}
    \subfile{files/chapter-2/powerschool.tex}

    \subsection{Esis Salama}\label{subsec:esis-salamae}
    \subfile{files/chapter-2/esis-salama.tex}

    \section{conclusion}\label{sec:conclusion}
    \subfile{files/chapter-2/conclusion.tex}

    \chapter{Conception}\label{sec:conception}
    \subfile{files/chapter-3/conception.tex}

    \section{Choix techniques et motivations}\label{sec:choix-techniques-et-motivation}
    \subsection{MySQL}\label{subsec:mysql}
    \subfile{files/chapter-3/mysql.tex}

    \subsection{Architecture de l'application}\label{sec:application-architecture}
    \subfile{files/chapter-3/application-architecture.tex}

    \subsection{UML}\label{sec:modeling-uml}
    \subfile{files/chapter-3/modeling-uml.tex}

    \subsection{Typescript}\label{sec:typescript}
    \subfile{files/chapter-3/typescript.tex}

    \subsection{NestJS}\label{sec:nestjs}
    \subfile{files/chapter-3/nestjs.tex}

    \subsection{NextJs}\label{sec:nextjs}
    \subfile{files/chapter-3/nextjs.tex}

    \subsection{Git}\label{sec:git}
    \subfile{files/chapter-3/git.tex}

    \subsection{GitHub}\label{sec:github}
    \subfile{files/chapter-3/github.tex}

    \subsection{Tailwindcss}\label{sec:tailwind}
    \subfile{files/chapter-3/tailwind.tex}

    \subsection{Fedora}\label{sec:fedora}
    \subfile{files/chapter-3/fedora.tex}

    \addcontentsline{toc}{subsection}{Conclusion}
    \subsection*{Conclusion}\label{sec:conclusion-choix-techniques}
    \subfile{files/chapter-3/conclusion.tex}

    \newpage
    \section{Modèles}\label{sec:database-models}
    \subsection{Diagramme de classes}\label{subsec:class-diagram}
    \subfile{files/chapter-4/class-diagram.tex}

    \subsection{Diagramme d'objets}\label{subsec:objects-diagram}
    \subfile{files/chapter-4/object-diagram.tex}

    \subsection{Diagramme de cas d'utilisation}\label{subsec:use-case-diagram}
    \subfile{files/chapter-4/use-case-diagram.tex}

    \subsection{Diagramme de séquence}\label{subsec:sequence-diagram}
    \subfile{files/chapter-4/sequence-diagram.tex}


\end{document}
